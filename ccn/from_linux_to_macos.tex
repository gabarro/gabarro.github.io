\title{from linux to macos}

I have been using linux and bsd since quitting win95 in 1997.  In 2024 I
acquired a mac mini that came with macos (darwin~23.5.0 according to
uname) and started using it sporadically.  These are my personal notes of the
experience, written as a reminder of the operations required to turn macos
into a system usable by plain graybeards.

Notice that this is just a test machine for verifying that my programs
compile and run correctly in a standard mac install.  I have no intention for
it to become my main system.  If there was a practical way to run macos
virtual machines on linux I'd not have bothered with a physical mac at all.

\section*{Zeroth day: the computer}

The computer is an ARM mac mini M2 with minimal features (8Go RAM, 256Go
SSD).  It is sold in 2024 for about 600 EUR.

The packaging is no-nonsense and easy to open.  Inside you find a cute
computer in a rounded aluminium box and a power chord.  The box has all the
typical ports: rj45, hdmi, headphone jack and four USB-A.  There is a
power button in the back and a tiny white led in the front to signal that
the computer is on.

I plug a keyboard, a mouse and a screen and switch it on.  I am greeted
immediately by a beautiful welcome screen and a discrete chime sound.
Everything looks alright, so I switch it off to continue next day.

\section*{First day: onboarding}

You don't need to install anything since the computer already comes with an
OS inside.  Still, before using it, you are asked to select your keyboard,
language, time zone and to create a user.  The process is very straightforward
and essentially identical to installing a linux distribution from a usb key.

Compared to a regular linux install (I have recent experience with ubuntu and
mint), there are a few improvements:

\begin{enumerate}
	\item The language is much more reassuring: you are never afraid to
		break anything or not being able to go back and fix stuff.
	\item The user interface is really well-polished.  For example, the
		keyboard configuration that you selected is always visible,
		and you can change it easily.  Especially during password
		entry.  I recall that on my last ubuntu install I had some
		trouble when entering my password because there was no
		indication of which keyboard map selected (I had chosen both
		qwerty and azerty to be able to switch easily between them).
		You get the feeling that the whole procedure has been
		battle-tested against a bunch of people that tried hard to
		make it fail, and that most of the ``holes'' are covered.
\end{enumerate}

%Some other things are equally bad:

There are also some things that are worse:

\begin{enumerate}
	\item Dark patterns to force you to sign-in to user-engagement and
		data-stealing programs: please enter your apple id! you don't
		have one? oh, please, open an apple account!  You need Siri!
		Sign-up to get siri!  Please, please, please!  You are asked
		that shit at least ten times before you can even run a
		terminal.
	\item If you are not connected to internet you are constantly
		pestered to setup your wifi, plug your ethernet, etc.
\end{enumerate}

At the end, if you are willing to fight, you can close all the spam windows
and get a working system where you can run terminal commands.  There are
some unix utilities but no C compiler.  More about this later.

Other things are equally bad in both systems.  For example, I was happy to
see that catalan was proposed as a localization language.  I selected it, but
the translation was incomplete and comically bad.  Some words were in
english, some words were, strangely in other languages.  Worse, the use of
the language does not correspond to the standard catalan usage
found elsewhere in computers.  These are subtle differences like using
infinitives instead of
imperatives for actions, using ``archive'' when it should use ``file'', and
so on.  Overall it feels like an unprofessional localization work, most
probably automated and not checked by anybody.  While this can be expected
from amateur-driven localizations as in linux, I was surprised to see this
lack of seriousness on a system that you pay for.

After a few minutes of use, the errors in the catalan localization became so
unbearable that I switched to english.  Strangely, some words in the
interface are still in catalan (names of colors, and such).  Even after a
reboot.

At least, in linux I do not have to endure the most glaring localization
errors: it's easy to download the source code and fix the .po files in you
computer.  Also, you can report the fixes to the developers and they will get
fixed for everybody in a next version of the software.  Under macos, this
kind of collaboration between users is simply not possible.
It would seem that it is actively discouraged, even.

Once I have created my user, logged in and out, and opened a web browser to
check that everything works, I switch it off for the next day.


\section*{Second day: unix}

A friend of mine promised me that macos provides a clean unix experience.  I
am thus surprised that there is no visible terminal icon in the desktop
interface.  There are more than twenty cute icons, but none of them happens
to open a terminal.

Most of that day was spent in adding this icon to the launcher,
removing all the other icons, and generally turning the user interface to a
more familiar one.

The graphical settings menu is very well-organized.  The gnome-control-center
is a sad joke next to it.   I assume that all settings can be configured from
the command line (editing configuration files or launching specific
programs), but the settings program is so beautiful and complete that I
cannot resist spending a few hours  bla bla bla . . . most of the work
consist in unclicking the "siri support" from all options

- more seriously, inside the terminal...

- file|cut|sort|uniq|sort ... "...find it funny that in 2024 macos is still a
perl shop", comment about bin PATH unification

- severely outdated unix tooling

- top ?

- ugly homedir, uppercase dirs

- some rant about uppercase/lowercase confusion in filenames

- xcode-install

- macos-specific stuff: sw_vers, software_update, etc

- 



\section*{Third day: sshd}

- activate it in the interface

- fake manpages, undocumented features

- unexplained caching delays, blocks

%\section*{Fourth day: ipad}





\section*{Fourth day: notarization}

- stallmanic explanation of the problem

- the "give permission to the terminal app to access your files" bullshit

- ancillary verification of the notarization by comparing running times of
scripts that compile stuff

- verification by looking at the network traffic




\section*{Fifth day: homebrew}

- discussion about macports/brew, etc.  we chose brew because it is more
popular, although ports seems better (and in spite of brew non-support for
older systems, cute naming, and general enshitiffication)

- set up user "brewer", the only one who can install brew stuff.

- explanation of cellar shit

- example setup that allows for universal makefiles that compile in linux
and mac using libpng

- some rant about gcc calling clang, and apple treating its users as adults
(but hey, we force you to use 20-year old unix tools because fuck you)



\section*{Sixth day: opengl, cocoa}

- glut still there, but with stupid deprecation warnings

- RGFW-like portability

- prepare for an even uglier future if wayland becomes more popular


\section*



% vim:set tw=77 filetype=tex spell spelllang=en:
