\title{from linux to macos}

I have been using linux and bsd since quitting win95 in 1997.  In 2024 I
acquired a mac mini that came with macos (darwin~23.5.0 according to
uname) and started using it sporadically.  These are my personal notes of the
experience, written as a reminder of the operations required to turn macos
into a system usable by plain graybeards.

Notice that this is just a test machine for verifying that my programs
compile and run correctly in a standard mac install.  I have no intention for
it to become my main system.  If there was a practical way to run macos
virtual machines on linux I'd not have bothered with a physical mac at all.

\section*{Zeroth day: the computer}

The computer is an ARM mac mini M2 with minimal features (8Go RAM, 256Go
SSD), which is sold in 2024 for about 600 EUR.

The packaging is no-nonsense and easy to open.  Inside you find a cute
computer in a rounded aluminium box and a power chord.  The box has all the
typical ports: rj45, hdmi, headphone jack and four USBs.  There is one
power button in the back and a tiny white in the front to signal that
the computer is on.

I plug a keyboard, a mouse and a screen and switch it on.  I am greeted
immediately by a beautiful welcome screen and a discrete chime sound.
Everything looks alright, so I switch it off to continue next day.

\section*{First day: onboarding}

You don't need to install anything since the computer already comes with an
OS inside.  Still, before using it, you need to select your keyboard,
language, time zone and create a user.  The process is very straightforward
and essentially identical to installing a linux distribution from a usb key.

Compared to a regular linux install (I have recent experience with ubuntu and
mint), there are a few improvements:

\begin{enumerate}
	\item The language is much more reassuring: you are never afraid to
		break anything or not being able to go back and fix stuff.
	\item The user interface is really well-polished.  For example, the
		keyboard configuration that you selected is always visible,
		and you can change it easily.  Especially during password
		entry.  I recall that on my last ubuntu install I had some
		trouble when entering my password because there was no
		indication of the keyboard map selected (I had chosen qwerty
		and azerty to be able to switch easily between them).
		You get the feeling that the whole procedure has been
		battle-tested against a bunch of people that tried hard to
		make it fail, and that most of the ``holes'' are covered.
\end{enumerate}

%Some other things are equally bad:

There are also some things that are worse:

\begin{enumerate}
	\item Dark patterns to force you to sign-in to user-engagement
		programs: please enter your apple id! you don't have one? oh,
		please, open an apple account!  Please, please, please!  You
		are asked that shit at least ten times before you can run a
		terminal.
	\item If you are not connected to internet you are constantly
		pestered to setup your wifi, plug your ethernet, etc.
\end{enumerate}

Other things are equally bad in both systems.  For example, I was happy to
see that catalan was proposed as a localization language.  I selected it, but
the translation was incomplete and comically bad.  Some words were in
english, some words were, strangely in other languages.  Worse, the use of
the language does not correspond to the traditional standard catalan used
elsewhere.  These are subtle differences like using infinitives vs.
imperatives for actions, using ``archive'' when it should use ``file'', and
so on.  Overall it feels like an unprofessional localization work, most
probably automated and not checked by anybody.  While this can be expected
from amateur-driven localizations like in linux, I was surprised to see this
lack of seriousness on a system that you are supposed to pay for.

After a few minutes of use, the errors in the catalan localization became so
unbearable that I switched to english.  Strangely, some words in the
interface are still in catalan (names of colors, and such).

At least, in linux I do not have to endure the most glaring localization
errors: it's easy to download the source code and fix them for you computer.
Also, you can report them to the developers and they will get fixed for
everybody in a next version of the software.  Under macos, this kind of
collaboration between users  is simply not possible.


\section*{Second day: unix}

\section*{Third day: sshd}

\section*{Fourth day: homebrew}



% vim:set tw=77 filetype=tex spell spelllang=en:
