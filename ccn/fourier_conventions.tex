
\newcommand{\ud}{\mathrm{d}}
\newcommand{\R}{\mathbf{R}}

\title{Fourier normalization conventions}

{\bf TLDR: } we argue that the best normalization convention for the Fourier
transform is~$\widehat{f}(y)=\tfrac{1}{\sqrt{2\pi}}\int f(x)e^{-ixy}\ud x$
because it is unitary and it does not require~$2\pi$ factors in the
derivative formulas.

\section{Definition and basic properties}

Let us fix three positive constants~$\alpha,\alpha',\beta$.  We define the
Fourier direct and inverse transforms as

$$
\widehat{f}\left(y\right)=\alpha\int f(x) e^{-i\beta xy}\ud x
$$
$$
\check{f}\left(y\right)=\alpha'\int f(x) e^{i\beta xy}\ud x
$$

The typical conventions are:

\begin{enumerate}
	\item The unitary convention~$\beta=1,\
		\alpha=\alpha'=\frac{1}{\sqrt{2\pi}}$
	\item The mathematical convention~$\beta=\alpha=1,\
		\alpha'=\frac{1}{2\pi}$
	\item The engineering convention~$\beta=2\pi,\ \alpha=\alpha'=1$.
\end{enumerate}

In the following list of properties we abuse a bit the notation and we
write~$f(x)$ to mean the function~$x\mapsto f(x)$.  Thus, for example,
instead of~$\widehat{f}(y)$ we write~$\widehat{f(x)}(y)$ and so on.

\begin{enumerate}
	\item Shift: $\widehat{f(x-a)}(y)=e^{-i\beta ay}\widehat{f}(y)$
	\item Zoom: $\widehat{f(x/a)}(y)=a\widehat{f}(ay)$
	\item Derivative inside: $\widehat{f'\,}(y)=\beta i y\,\widehat{f}(y)$
	\item Derivative outside: $\widehat{f}(y)'=-\beta i
		y\,\widehat{xf(x)}(y)$
	\item Convolution:
		$\widehat{f*g}=\frac{1}{\alpha}\widehat{f}\,\widehat{g}$
	\item Product:
		$\widehat{fg}=\widehat{f}*\widehat{g}$
	\item Plancherel:
		%$\int f\overline{g}=\int\widehat{f}\overline{\widehat{g}}$
		$\left<f,g\right>=\mu(\alpha)\left<\widehat{f},\widehat{g}\right>$
	\item Parseval:
		$\left\|f\right\|_{L^2(\R)}=\mu(\alpha)\left\|\widehat
		f\right\|_{L^2(\R)}$
		%$\int f\overline{g}=\int\widehat{f}\overline{\widehat{g}}$
	\item Gaussian:
		$\widehat{e^{-x^2}}(y)=\alpha\sqrt{\pi}e^{-\beta y^2/4}$\\
	\item Gaussian:
		$\widehat{e^{-x^2/2\sigma^2}}(y)=\sigma\alpha\sqrt{2\pi}e^{-\beta
		\sigma^2y^2/4}$\\
\end{enumerate}

% vim:set tw=77 filetype=tex spell spelllang=en:
