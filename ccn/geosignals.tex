\title{Geometric Signals}



\newcommand{\1}{\mathbf{1}}
\newcommand{\R}{\mathbf{R}}
\newcommand{\T}{\mathbf{T}}
\newcommand{\Z}{\mathbf{Z}}

To study a geometric object~$M$, people often use regular functions
between~$M$ and spaces with well-known properties, such as~$\R^n$,~$\T^n$
or~$S^n$.  The resulting spaces of functions have an algebraic structure
whose properties correspond to geometric properties of~$M$.
Thus, the number of connected components, number of holes, etc, arise as
dimensions of vector spaces or cardinals of groups.
Typically,
families of functions of the form~$X^n\to M$ constitute a sequence of groups
(homology, homotopy), and the structure of these groups and their quotients
 lot of geometrical information about~$M$.  On the other hand,
families of functions~$M\to X^n$ form a sequence of vector spaces
(cohomology) whose dimensions and quotients give geometrical information
about~$M$, in a different, often easier to interpret form.

The easiest example is the algebraic definition of~\emph{connexity}.  Imagine
that~$M$ has~$n$ connected components.  How can we recover~$n$ using this kind
of algebra?
%
Via cohomology: consider the set of smooth functions~$M\to\R$ whose
derivative is zero.  This set is a vector space of dimension~$n$.
Notice that this is the quotient space of the set of all smooth functions,
modulo its subspace defined by a linear condition~$df=0$.
%
Via homology: consider the free abelian group generated by all
paths~$[0,1]^1\to M$.  The boundary of such a path is defined as the difference
between its two endpoints, which is an element of the free group generated by
all points~$[0,1]^0\to M$.  The quotient of two such groups is isomorphic
to~$\Z^n$.

Functions~$\R\to M$ are thus called~\emph{paths} or~\emph{walks},
or~\emph{trajectories}.  And functions~$M\to\R$ will be called
called~\emph{signals}, or simply~\emph{functions}.

In this note we recall the main definitions of ``geometric signal
processing''. That is, we study the many operations that can be performed with
sets of functions of the form~$M\to X^n$.  We just give the definitions and
the main main results; no proofs nor lengthy interpretations.


\clearpage
\section{Vector calculus}

\subsection{vector calculus in the plane}

\subsection{vector calculus in three-dimensional space}

\subsection{integral theorems: green, gauss, stokes}

\subsection{limitations of vector calculus (e.g. two types of vector fields)}

\subsection{geometry of plane curves, space curves, surfaces}


\clearpage
\section{Differential geometry}

% GRAND SCHEME OF NON-METRIC STUFF, SUBSET OF NEXT SECTION'S SCHEME

\subsection{manifolds, charts, atlases}

The easiest way to think of a manifold is as~$\R^n$, but where most operations
are severely restricted: you cannot sum points, multiply points by scalars,
nor compute inner products.  There is no notion of translation, nor
symmetry.  What~\emph{can} you do, then?  You can only have smooth
curves~$\R\to M$ and functions~$M\to\R$, and relations between
these two objects.  Notice that if you have a curve~$\gamma:\R\to M$ and a
function~$f:M\to\R$ then the composition~$f\circ\gamma$ is a
function~$\R\to\R$ upon which it is straightforward to reason.

Formally, a differential manifold is defined as a topological space~$M$
together with a set of charts, that is, continuous bijective
functions~$\varphi:U\to\R^n$, where~$U$ are open sets of~$M$, such that for
any pair of charts~$\varphi,\varphi'$ the
function~$\varphi'\circ\varphi^{-1}$ is a smooth diffeomorphism of~$\R^n$
(nonempty if~$U\cap U'\neq\emptyset$).  A trivial example is~$\R^n$ itself,
with the identity function as a single chart.

An equivalent definition (but the equivalence is non-trivial) is that an
$n$-dimensional manifold is set of the form~$F^{-1}(\{0\})$
where~$F:\R^{n+k}\to\R^n$ is a smooth function such that~$0$ is not a
critical value.

A mapping~$f:M\to N$ between manifolds is said to be smooth when for any
charts~$\varphi:U\to\R^m$, $\psi:V\to\R^n$ the
mapping~$\psi\circ\varphi^{-1}:\R^m\to\R^n$ is smooth.
As particular cases  of this definition you have smooth curves and smooth
signals.

The vector space of smooth signals on~$M$ is denoted by~$C^\infty(M)$.

\subsection{tangent vectors, vector fields, tangent bundle}

There are several equivalent ways to define~\emph{tangent vectors}.
Let~$p$ be a point of a manifold~$M$.

(T1) A tangent vector~$X_p$ at the point~$p$ is an equivalence class of
curves~$\gamma:\R\to M$ through~$p$ (i.e., such that~$\gamma(0)=p$)
modulo the equivalence relation
\[
	\gamma_1\sim\gamma_2
	\iff
		(\varphi\circ\gamma_1)'(0)=(\varphi\circ\gamma_2)'(0)
\]
for all charts~$\varphi:U\to\R^n$ such that~$p\in U$.  Notice that this
definition makes sense because the maps~$\varphi\circ\gamma_i$ are smooth
functions~$\R^n\to\R^n$.

(T2) A tangent vector~$X_p$ at the point~$p$ is a linear derivation on the
ring~$C^\infty(M)$.  That is, an~$\R$-linear function~$X_p:C^\infty(M)\to\R$
such that
\[
	X_p(fg)=f(p)X_p(g) + g(p)X_p(f)
\]
for all~$f,g\in C^\infty(M)$.  The number~$X_p(f)$ is to be interpreted as
the directional derivative of the function~$f$ in the direction of the
vector~$X_p$.  Notice that an immediate consequence of the definition is
that~$X_p(c)=0$ for constant functions~$c$.

(T3) If~$\varphi$ is a local chart around~$p$, a tangent vector at~$p$ is
given by~$n$ numbers~$(x^1,\ldots,x^n)\in\R^n$, called ``the coordinates of
the vector in the chart~$\varphi$''.

For the last definition to make sense, we need a way to say whether vectors
given by coordinates in different charts are equal.  This happens when their
coordinates are linearly related by the jacobian matrix of the
map~$\varphi'\circ\varphi^{-1}:\R^n\to\R^n$.  To see in what order the
matrix product happens, look at the simplest example:~$M=\R^n$, one chart is
the identity map and the other one is an arbitrary diffeomorphism.

The~\emph{tangent space} of~$M$ at~$p$ is the set~$T_pM$ of all tangent
vectors at~$p$.  It happens to be a vector space of dimension~$n$.

A~\emph{vector field} is the assignment of a tangent vector to each point of
the manifold.  This definition is not very operational if we want to talk
about how regular a vector field is, etc.  The appropriate definition of
vector field is a section of the so-called~\emph{tangent bundle}~$TM$, that
must be defined first.  The tangent bundle is the union of all tangent
spaces~$TM=\bigcup_{p\in M}T_pM$.  It happens to be a manifold of
dimension~$2m$, and it is equipped with the natural projection
operator~$\pi:TM\to M$ that recovers the base point of each
vector:~$\pi\left(X_p\right)=p$.  There are like seven different definitions
of the tangent bundle, and Spivak's monograph spends a large amount of the
first volume to prove that all these definitions are equivalent.

Here, we recall the following definitions of a vector field.  Definitions
F1,F2,F3 mirror the definitions T1,T2,T3 of a tangent vector, but F0 is
different:

(F0) A vector field is a section of the tangent bundle, that is a
smooth map~$X:M\to TM$ such that~$\pi\circ X$ is the identity on~$M$.

(F1) A vector field is given by a local flow field, i.e., a smooth
function~$\phi:M\times\R\mapsto M$ such that~$\phi(p,0)=p$.

(F2) A vector field is an~$\R$-linear derivation, i.e. a linear
map~$X:C^\infty(M)\to C^\infty(M)$ such that
\[
	X(fg)=fX(g)+gX(f)
\]
for all~$f,g\in C^\infty(M)$.

(F3) A vector field is given on a coordinate chart by a set of~$n$
functions~$a^i:\R^n\to\R$ called the components of the field on that
coordinate system.  Changes of coordinates follow the same law as in (T3).

The set of all vector fields is denoted~$\mathcal{X}(M)$.
It is an~$\R$-vector space and a~$C^\infty(M)$-module.

\subsection{tensors, tensor product, contraction}

\subsection{differential forms, wedge product}

\subsection{exterior derivative, closed forms, betti numbers}

\subsection{chains, integrals and stokes theorem}

\subsection{currents and forms}

\subsection{lie derivatives (of functions, forms, tensors)}

\subsection{lie bracket}


\clearpage
\section{Riemannian geometry}

% GRAND SCHEME OF METRIC STUFF WITH METRIC-ONLY STUFF IN RED

\subsection{metric, length, energy}

\subsection{geodesic equations}

\subsection{musical isomorphisms, hodge duality}

\subsection{laplace beltrami}

\subsection{covariant derivative}

\subsection{parallel transport}

\subsection{killing vector fields (infinitessimal isometries)}

\subsection{hessian}

\subsection{curvatures}


\clearpage
\section{Other structured geometries}

\subsection{symplectic structure}

poisson bracket, volume form, cotangent bundle

\subsection{complex structure}

\subsection{kahlerian structure}


\clearpage
\section{Discrete case}



% vim:set tw=77 filetype=tex spell spelllang=en sw=2 ts=2:
