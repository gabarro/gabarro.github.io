\title{Math notes in music theory}

These are some personal notes on the mathematical theory of music.
The required knowledge is single-variable calculus and Fourier analysis.

Overview: A musical piece is a single sound wave.  This sound wave can be
analyzed at different levels.  The four levels, in increasing order of
complexity are timbre, harmony, rhythm and melody.

\section{Timbre}

If you play the same note in different instruments it sounds
different.  This difference is called~\emph{timbre}.
%The timbre is the most important aspect in music theory, for it will
%determine the harmony and the melody.
Let us give a mathematical formalization of timbre.

\subsection{Definition of timbre}
A sound wave is a function~$f(t)$ that describes the evolution of air
pressure along the time~$t$.  To define the timbre of~$f(t)$ we assume first
that the sound is in a stationary regime: thus we ignore the attack of the note
and its decay over time; we will review this simplification later.  This means
that the sound wave can be expressed in the following form
\begin{equation}\label{eq:soundwave}
	f(t) = \sum_{n\ge 1} a_n\sin\left(\omega_n t\right)
\end{equation}
where~$a_n>0$ and~$\omega_n>0$ for~$n=1,2,3,\ldots$.  For all practical
purposes we can assume that this sum has a finite number of terms.  The
number~$\omega_1$ is called the~\emph{fundamental frequency} or
the~\emph{pitch} of~$f(t)$, and the numbers~$\mu_n=\dfrac{\omega_n}{\omega_1}$
are called the~\emph{overtones} of the sound~$f(t)$.  The numbers~$a_n$ are
called the~\emph{amplitudes} and the numbers~$\dfrac{a_n}{a_1}$ are
the~\emph{relative amplitudes} of the sound~$f(t)$.  Either~$a_1$
or~$\sqrt{\sum {a_n}^2}$ is called the~\emph{volume} of the
sound.  For many typical sounds the sequence of amplitudes is
non-increasing~$a_n\ge a_{n+1}$ and it can be normalized by setting~$a_1=1$.

Thus the definition of~\emph{timbre} of a stationary sound~$f(t)$ as in
formula~(\ref{eq:soundwave}) is the list of its
overtones~$\mu_n=\frac{\omega_n}{\omega_1}$ and its relative
amplitudes~$\frac{a_n}{a_1}$.  We can obtain sounds of that same timbre at
arbitrary pitches and volumes by changing~$\omega_1$ and~$a_1$ and keeping
the same overtones and relative amplitudes.


This definition warrants some commentary.  The first comment is, the fact
that a sound wave is stationary does not mean that it is periodic.  For
example, the function~$f(t)=\sin(t)+\sin(\pi t)$ is not periodic but it is
nonetheless stationary.  Hearing the sound~$f(t)$, one perceives the
superposition of two pure waves of different frequency.  It is
indistinguishable from the sound
wave~$g(t)=\sin(t)+\sin\left(\tfrac{22}7t\right)$ which is periodic.
Periodicity is a mostly irrelevant characteristic of~$f(t)$.

The second comment, closely related to the first one, concerns the phases of
the sinusoidal waves.  We could have defined a stationary sound wave as
\begin{equation}\label{eq:soundwave-phases}
	f(t)=\sum_{n\ge 0}a_n\sin(\omega_nt+\varphi_n)
\end{equation}
or equivalently
\begin{equation}\label{eq:soundwave-cosines}
	f(t)=
	\sum_{n\ge 0}a_n\sin(\omega_nt)
	+\sum_{n\ge 1}b_n\cos(\omega_nt).
\end{equation}
Notice that the forms~(\ref{eq:soundwave-cosines})
and~(\ref{eq:soundwave-phases}) are equivalent through elementary trigonometry.
Auditory experiences by Ohm (1843) show that the phases~$\varphi_n$
have no audible effect.  The reality is of course more subtle, and in some
cases phase distortion is indeed audible (see
\url{http://www.silcom.com/~aludwig/Phase_audibility.htm} and the references
therein), but for the moment we will assume that the observation by Ohm holds
true.  Thus, the phases~$\varphi_n$ on formula~\ref{eq:soundwave-phases} can
take arbitrary values, and for all we care we set them to~$\varphi_n=0$,
recovering formula~\ref{eq:soundwave}.

Using a digital synthesizer one can easily produce sounds of the general
form~\ref{eq:soundwave} above.  Any particular choice of overtones and their
amplitudes will give a particular timbre; and that's all there is to timbre.

% harmonic vs inharmonic waveforms
\subsection{Harmonic and inharmonic waveforms}

In the very particular case that the overtones are integer multiples of the
fundamental~(i.e. that ~$\omega_n=n\omega_1$, or equivalently that~$\mu_n=n$)
the sound wave is called~\emph{harmonic}.  If there are some overtones that
are not integer multiples of the fundamental, they are
called~\emph{partials}, and the sound is called~\emph{inharmonic}.
Here are some notes:~\lilypond{a b c} bla bla bla.

%\begin{abc}
%X: 1
%K: C
%Ccge'
%\end{abc}

% typical waveforms
\subsection{Typical waveforms}

% decay
\subsection{Decay}

% attack
\subsection{Attack}

% phases revisited
\subsection{Phases}

% convolution theorem!
\subsection{Localization and formants}

% physical modelling of timbre
\subsection{Physical modelling of timbre}

% detailed study of the bell
\subsection{Detailed study of the bell}


\section{Harmony}

% dissonance
\subsection{Dissonance}

% harmonic consonance
\subsection{Harmonic consonance}

% chords
\subsection{Chords}

% chord sequences
\subsection{Chord sequences}

% scales, tuning, and the circle of fifths
\subsection{Scales and tuning}

% 
\subsection{The circle of fifths}

\subsection{Exotic octave-based scales}

\subsection{Exotic non-octave-based scales}

\section{Rhythm}

Rhythm is low-frequency timbre.
It is so low frequency that we can perceive the individual oscillations of
the terms.  As opposed to the case of timbre, the phases of each repeating
pattern are very important.  The period is also a fundamental characteristic,
called the~\emph{measures}.

% see sethares rhythm book...

\section{Melody}

Melody is the most difficult part of music theory to formalize in our
framework.
A melody can be formalized naively as a path in note space or,
if it has more structure, as a ``low-frequency harmony``.  Let us develop
this definition.
% melody as a path in note space

\section*{References}

* Music: a Mathematical Offering, Dave Benson. Explains the mathematical
theory of how to compute the timbre of an instrument given its shape. Also
how the timbre of the instruments determines harmony.

* Tuning, timbre, spectrum, scale, William Sethares. A deep exploration of
the relationship between timbre and harmony. It starts by exhibiting a
(synthetic) instrument whose octaves are dissonant, and the music that you
can do with it. Then it doubles down on this idea to obtain a lot of fun.

* Rhythm and Transforms, William Sethares. A mathematical theory of rhythm.

* The Topos of Music, Guerino Mazzola. Really hardcore mathematical music
theory, too scary for simple people like me.

* \url{https://raw.githack.com/CindyJS/ScaleLab/master/index.html} and more
generally
\url{https://www.imaginary.org/exhibition/la-la-lab-the-mathematics-of-music}.


% vim:set tw=77 filetype=tex spell spelllang=en:
