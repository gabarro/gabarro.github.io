


\newcommand{\1}{\mathbf{1}}
\newcommand{\R}{\mathbf{R}}
\newcommand{\T}{\mathbf{T}}
\newcommand{\Z}{\mathbf{Z}}
\newcommand{\ud}{\mathrm{d}}
\newcommand{\ds}{\displaystyle}

% \abs{x}         ->    |x|
% \Abs{x}         ->   ||x||
% \ABS{x}         ->  |||x|||
\newcommand{\abs}[1]{\left|#1\right|}
\newcommand{\Abs}[1]{\left\|#1\right\|}
\newcommand{\ABS}[1]{{\left\vert\kern-0.25ex\left\vert\kern-0.25ex\left\vert #1 \right\vert\kern-0.25ex\right\vert\kern-0.25ex\right\vert}}

% \parens{x}      ->  (x)
% \pairing{x}{y}  ->  <x,y>
\newcommand{\parens}[1]{\left(#1\right)} % (x)
\newcommand{\pairing}[2]{\left\langle #1,\,#2\right\rangle} % <x,y>




\title{géométrie différentielle}

\begin{center}
proposition de cours pour le M1 Hadamard
\end{center}

\paragraph{Motivation.}
Une partie importante des normaliens du département de mathématiques
finissent leur parcours sans avoir jamais vu les équations d'Euler-Lagrange
ni les notions de courbure d'une courbe ou d'une surface.  Même parmi ceux
qui passent l'agrégation.  Il y avait~\emph{une seule} leçon de courbes et
surfaces dans le programme d'agrégation, mais en 2024 elle a été éliminée.
Ceci est particulièrement agaçant lorsqu'il se passe dans le pays qui a vu
fleurir les travaux de Lagrange, Darboux, Bonnet, Cartan, de Rham, Berger,
Colin de Verdière, \ldots.  L'objectif de ce cours est de corriger localement
cette triste tendance géométricide.

\paragraph{Philosophie.}
C'est d'abord un cours ``classique'' où l'on traite en détail les définitions
et les preuves théorèmes principaux.  Cependant, l'approche est plutôt à là
russe qu'à la française: on annonce les résultats dans~\emph{la moindre}
généralité possible qui les rend encore intéressants.  L'objectif n'est
surtout pas d'introduire un langage général, mais résoudre des problèmes
géométriques concrets, qui nécessitent souvent une partie de ce langage.
Ainsi, on voit d'abord le trièdre de Serret-Frenet, et plus tard, quand on en
a besoin, les fibrés sur les variétés.

\paragraph{Thèmes.}
Il y a des thèmes récurrents qui guident le défilé du cours.

D'abord, les \emph{méthodes variationnelles} sont au coeur du sujet: on
caractérise les droites comme les courbes de longueur minimale dans l'espace
euclidien, puis les géodésiques comme des courbes de longueur minimale sur
une variété, ensuite les surfaces minimales, etc.  Le fleuron final du cours
est la caractérisation de l'équation de champ d'Einstein comme l'équation
d'Euler-Lagrange de l'action de Hilbert, définie à partir de la courbure de
Ricci.  Évidemment la physique, spécialement la mécanique Lagrangienne, est
un sujet toujours présent dans les exemples et dans les exercices.

Le deuxième thème est~\emph{le lien continu-discret}.  Surtout dans les
exercices, on développe beaucoup d'analogies entre la géométrie
différentielle et les modèles discrets comme les graphes et les complexes
cellulaires.  Grâce au formalisme des courants de de Rham, on verra que
cette analogie permet de voir souvent les modèle discrets comme un cas
particuliers de modèles continus.

Le troisième thème, un peu caché mais toujours en arrière-plan, est l'algèbre
homologique.  Pour étudier un objet géométrique~$X$, on le fait souvent de
manière indirecte, en étudiant des ensembles de fonctions~$\R^n\to X$, qui
forment des groupes (homologie, groupes fondamentaux, etc) ou des ensembles
de fonctions~$X\to\R^n$, qui forment des espaces vectoriels (cohomologie,
etc).  Les suites de Mayer-Vietoris ne sont jamais mentionnées par ce nom,
mais elles font leur apparition par exemple quand on formalise le calcul
vectoriel en~$\R^3$ avec un langage géométrique.

\paragraph{Organisation.}
Le cours est prévu pour avoir lieu entre 8 et 12 semaines, à teneur de 1.5
heures de cours par semaine, plus 2.5 heures d'exercices.  Le cours a deux
moitiés: la première moitié porte sur les courbes et surfaces, et la deuxième
sur la géométrie riemannienne générale.  Les leçons sont aussi indépendantes
que possible, et dans chaque leçon on annonce et on démontre un résultat
célèbre.

\paragraph{Contexte.}
Ce cours n'a presque pas intersection avec celui de Paulin.  Par contre, il
est très similaire à l'ancien cours de Pansu (avec un peu moins de physique).

\paragraph{L'enseignant.}
Je ne suis pas du tout spécialiste en géométrie différentielle (quoique, une
partie de ma thèse portait sur les inégalités isopérimétriques en variétés
Finslériennes, et j'ai publié plusieurs articles sur des modèles
variationnelles sur variétés, avec des applications au traitement d'image).
Je suis juste un utilisateur du langage, et je le maitrise bien.
J'ai toujours voulu enseigner un cours de géométrie différentielle.

\paragraph{Leçon 1.  Courbes planes: théorie locale}
Cette leçon sert à ouvrir l'appétit avec des exemples minimaux des sujets qui
seront abordés à plusieurs reprises dans la suite.  Ainsi, on caractérise les
droites comme les courbes paramétrées ou le fonctionnel de longueur
$$E\parens{\gamma}=\int \Abs{\dot\lambda(t)}\ud t$$
atteint son minimum.  L'équation d'Euler-Lagrange~$\frac{\delta
E}{\delta\gamma}$ amène à la notion de courbure



% vim:set tw=77 filetype=tex spell spelllang=fr sw=2 ts=2:
