\title{Geometric Signals}



\newcommand{\1}{\mathbf{1}}
\newcommand{\R}{\mathbf{R}}
\newcommand{\T}{\mathbf{T}}
\newcommand{\Z}{\mathbf{Z}}
\newcommand{\ud}{\mathrm{d}}
\newcommand{\ds}{\displaystyle}

To study a geometric object~$M$, people often use regular functions
between~$M$ and spaces with well-known properties, such as~$\R^n$,~$\T^n$
or~$S^n$.  The resulting spaces of functions have an algebraic structure
whose properties correspond to geometric properties of~$M$.
Thus, the number of connected components, number of holes, etc, arise as
dimensions of vector spaces or cardinals of groups.
Typically,
families of functions of the form~$X^n\to M$ constitute a sequence of groups
(homology, homotopy), and the structure of these groups and their quotients
 lot of geometrical information about~$M$.  On the other hand,
families of functions~$M\to X^n$ form a sequence of vector spaces
(cohomology) whose dimensions and quotients give geometrical information
about~$M$, in a different, often easier to interpret form.

The easiest example is the algebraic definition of~\emph{connexity}.  Imagine
that~$M$ has~$n$ connected components.  How can we recover~$n$ using this kind
of algebra?
%
Via cohomology: consider the set of smooth functions~$M\to\R$ whose
derivative is zero.  This set is a vector space of dimension~$n$.
Notice that this is the quotient space of the set of all smooth functions,
modulo its subspace defined by a linear condition~$df=0$.
%
Via homology: consider the free abelian group generated by all
paths~$[0,1]^1\to M$.  The boundary of such a path is defined as the difference
between its two endpoints, which is an element of the free group generated by
all points~$[0,1]^0\to M$.  The quotient of two such groups is isomorphic
to~$\Z^n$.

Functions~$\R\to M$ are thus called~\emph{paths} or~\emph{walks},
or~\emph{trajectories}.  And functions~$M\to\R$ will be called
called~\emph{signals}, or simply~\emph{functions}.

In this note we recall the main definitions of ``geometric signal
processing''. That is, we study the many operations that can be performed with
sets of functions of the form~$M\to X^n$.  We just give the definitions and
the main main results; no proofs nor lengthy interpretations.


\clearpage
\section{Vector calculus}

\subsection{vector calculus in the plane}

\subsection{vector calculus in three-dimensional space}

\subsection{integral theorems: green, gauss, stokes}

\subsection{limitations of vector calculus (e.g. two types of vector fields)}

\subsection{geometry of plane curves, space curves, surfaces}


\clearpage
\section{Differential geometry}

% GRAND SCHEME OF NON-METRIC STUFF, SUBSET OF NEXT SECTION'S SCHEME

\subsection{manifolds, charts, atlases}

The easiest way to think of a manifold is as~$\R^n$, but where most operations
are forbidden: you cannot sum points, multiply points by scalars,
nor compute inner products.  There is no notion of translation, nor
symmetry.  What~\emph{can} you do, then?  You can only have smooth
curves~$\R\to M$ and functions~$M\to\R$, and relations between
these two objects.  Notice that if you have a curve~$\gamma:\R\to M$ and a
function~$f:M\to\R$ then the composition~$f\circ\gamma$ is a
function~$\R\to\R$ upon which it is straightforward to reason.

Formally, a differential manifold is defined as a topological space~$M$
together with a set of charts, that is, continuous bijective
functions~$\varphi:U\to\R^n$, where~$U$ are open sets of~$M$, such that for
any pair of charts~$\varphi,\varphi'$ the
function~$\varphi'\circ\varphi^{-1}$ is a smooth diffeomorphism of~$\R^n$
(nonempty if~$U\cap U'\neq\emptyset$).  A trivial example is~$\R^n$ itself,
with the identity function as its only chart.

An equivalent definition (but the equivalence is non-trivial) is that an
$n$-dimensional manifold is set of the form~$F^{-1}(\{0\})$
where~$F:\R^{n+k}\to\R^n$ is a smooth function such that~$0$ is not a
critical value.

A mapping~$f:M\to N$ between manifolds is said to be smooth when for any
charts~$\varphi:U\to\R^m$, $\psi:V\to\R^n$ the
mapping~$\psi\circ\varphi^{-1}:\R^m\to\R^n$ is smooth.
As particular cases  of this definition you have smooth curves and smooth
signals.

The ring of smooth signals on~$M$ is denoted by~$C^\infty(M)$.

\subsection{tangent vectors, vector fields, tangent bundle}

There are several equivalent ways to define~\emph{tangent vectors}.
Let~$p$ be a point of a manifold~$M$.

(T1) A tangent vector~$X_p$ at the point~$p$ is an equivalence class of
curves~$\gamma:\R\to M$ through~$p$ (i.e., such that~$\gamma(0)=p$)
modulo the equivalence relation
\[
	\gamma_1\sim\gamma_2
	\iff
		(\varphi\circ\gamma_1)'(0)=(\varphi\circ\gamma_2)'(0)
\]
for all charts~$\varphi:U\to\R^n$ such that~$p\in U$.  Notice that this
definition makes sense because the maps~$\varphi\circ\gamma_i$ are smooth
functions~$\R^n\to\R^n$.

(T2) A tangent vector~$X_p$ at the point~$p$ is a linear derivation on the
ring~$C^\infty(M)$.  That is, an~$\R$-linear function~$X_p:C^\infty(M)\to\R$
such that
\[
	X_p(fg)=f(p)X_p(g) + g(p)X_p(f)
\]
for all~$f,g\in C^\infty(M)$.  The number~$X_p(f)$ is to be interpreted as
the directional derivative of the function~$f$ in the direction of the
vector~$X_p$.  Notice that an immediate consequence of the definition is
that~$X_p(c)=0$ for constant functions~$c$.

(T3) If~$\varphi$ is a local chart around~$p$, a tangent vector at~$p$ is
given by~$n$ numbers~$(x^1,\ldots,x^n)\in\R^n$, called ``the coordinates of
the vector in the chart~$\varphi$''.

For the last definition to make sense, we need a way to say whether vectors
given by coordinates in different charts are equal.  This happens when their
coordinates are linearly related by the jacobian matrix of the
map~$\varphi'\circ\varphi^{-1}:\R^n\to\R^n$.  To see in what order the
matrix product happens, look at the simplest example:~$M=\R^n$, one chart is
the identity map and the other one is an arbitrary diffeomorphism.

The~\emph{tangent space} of~$M$ at~$p$ is the set~$T_pM$ of all tangent
vectors at~$p$.  It happens to be a vector space of dimension~$n$.

A~\emph{vector field} is the assignment of a tangent vector to each point of
the manifold.  This definition is not very operational if we want to talk
about how regular a vector field is, etc.  The appropriate definition of
vector field is a section of the so-called~\emph{tangent bundle}~$TM$, that
must be defined first.  The tangent bundle is the union of all tangent
spaces~$TM=\bigcup_{p\in M}T_pM$.  It happens to be a manifold of
dimension~$2m$, and it is equipped with the natural projection
operator~$\pi:TM\to M$ that recovers the base point of each
vector:~$\pi\left(X_p\right)=p$.  There are like seven different definitions
of the tangent bundle, and Spivak's monograph spends a large amount of the
first volume to prove that all these definitions are equivalent.

Here, we recall the following definitions of a vector field.  Definitions
F1,F2,F3 mirror the definitions T1,T2,T3 of a tangent vector, but F0 is
different:

(F0) A vector field is a section of the tangent bundle, that is a
smooth map~$X:M\to TM$ such that~$\pi\circ X$ is the identity on~$M$.

(F1) A vector field is given by a local flow field, i.e., a smooth
function~$\phi:M\times\R\mapsto M$ such that~$\phi(p,0)=p$.

(F2) A vector field is an~$\R$-linear derivation, i.e. a linear
map~$X:C^\infty(M)\to C^\infty(M)$ such that
\[
	X(fg)=fX(g)+gX(f)
\]
for all~$f,g\in C^\infty(M)$.

(F3) A vector field is given on a coordinate chart by a set of~$n$
functions~$a^i:\R^n\to\R$ called the components of the field on that
coordinate system.  Changes of coordinates follow the same law as in (T3).

The set of all vector fields is denoted~$\mathcal{X}(M)$.
It is an~$\R$-vector space and a~$C^\infty(M)$-module.

The canonical basis of~$T_p\R^d$
is~$\displaystyle\left\{\left.\frac{\partial}{\partial
	x^i}\right|_p\right\}_{i=1,\ldots,d}$.  Any vector
	field~$X\in\mathcal{X}(\R^d)$ can thus be written uniquely as
\[
	X =
	a^1\frac{\partial}{\partial x^1}
	+\cdots+
	a^d\frac{\partial}{\partial x^d}
\]
where~$a^1,\ldots,a^d\in C^\infty(M)$.

\subsection{tensors, tensor product, contraction}

%SCRIPT gnuplot >gs_p.png <<EOF
%SCRIPT set term pngcairo size 320,200 enhanced font "BaskervaldADFStd,22"
%SCRIPT set nokey ; set notics ; set noborder
%SCRIPT set label "{/:Italic p}" at 0.5,-0.5
%SCRIPT plot [-2:3] [-2:3] '-' using 1:2 w points pt 7 ps 2 lc rgb 'black'
%SCRIPT      0 0
%SCRIPT      e
%SCRIPT EOF

%SCRIPT gnuplot >gs_Xp.png <<EOF
%SCRIPT set term pngcairo size 320,200 enhanced font "BaskervaldADFStd,22"
%SCRIPT set nokey ; set notics ; set noborder
%SCRIPT set arrow 1 from 0,0 to 3,2 filled
%SCRIPT set label "X_{/:Italic p}" at 3.5,1.5
%SCRIPT set label "{/:Italic p}" at 0.5,-0.5
%SCRIPT plot [-1:4] [-1:4] '-' using 1:2 w points pt 7 ps 2 lc rgb 'black'
%SCRIPT      0 0
%SCRIPT      e
%SCRIPT EOF

%SCRIPT gnuplot >gs_wp.png <<EOF
%SCRIPT set term pngcairo size 320,200 enhanced font "BaskervaldADFStd,22"
%SCRIPT set nokey ; set notics ; set noborder
%SCRIPT set lmargin 0
%SCRIPT set tmargin 0
%SCRIPT set bmargin 0
%SCRIPT set tmargin 0
%SCRIPT set label '' at 0,0 left offset char 2,0 point pt 7 ps 2 lc 0
%SCRIPT set label "ω_{/:Italic p}" at 1.9,1.2
%SCRIPT set label "{/:Italic p}" at -0.05,-0.5
%SCRIPT unset surface
%SCRIPT set view map
%SCRIPT unset table
%SCRIPT set contour base
%SCRIPT set cntrparam levels 13
%SCRIPT set for [i=1:8] linetype i linecolor 0
%SCRIPT splot [-2:2] [-2:2] 15*x-2*y
%SCRIPT EOF

\begin{tabular}{ccc}
	\includegraphics[width=0.3\linewidth]{gs_p.png} &
	\includegraphics[width=0.3\linewidth]{gs_Xp.png} &
	\includegraphics[width=0.3\linewidth]{gs_wp.png} \\
	A point~$p$ &
	A tangent vector~$X_p$ &
	A tangent covector~$\omega_p$
\end{tabular}

Let~$V$ be a vector space.  Elements of~$V$ are called~\emph{vectors} and
elements of its dual~$V^*$ are called~\emph{covectors}.
Thus if~$X\in V$ and~$\omega\in V^*$ then~$\omega(X)$ is a scalar.
You can think of~$V$ as ``column vectors'' and~$V^*$ as ``row vectors'', but
this is a somewhat limited view.

A~\emph{tensor of type~$(r,s)$} is a by definition multilinear map
\[
	\left(V^*\right)^r\times V^s\to\R.
\]
Thus a scalar is a tensor of type~$(0,0)$, a vector is a tensor of
type~$(1,0)$, a covector is a tensor of type~$(0,1)$, a linear map is a
tensor of type~$(1,1)$ and a bilinear form is a tensor of type~$(0,2)$.
If~$V$ is~$d$-dimensional, the set of tensors of type~$(r,s)$ has
dimension~$d^{r+s}$.

The~\emph{tensor product} of two tensors~$T\otimes T'$ is defined in a natural
way and is a tensor of type~$(r+r',s+s')$.

By setting~$V=T_pM$ we obtain a family of tangent tensor spaces.  Those can
be lifted to~\emph{tensor fields}: we denote by~$\mathcal{T}^{r,s}(M)$ is the
set of all tensor fields of type~$(r,s)$ on~$M$.
Thus~$\mathcal{T}^{0,0}(M)=C^\infty(M)$
and~$\mathcal{T}^{1,0}=\mathcal{X}(M)$.

Recall Penrose graphical notation and abstract index notation using Einstein
summation.

Tangent vectors at~$p$: the velocity of a curve~$R\to M$
through~$p$.  Tangent covectors: the gradient of a function~$M\to\R$ at~$p$.

If~$X$ is a vector field and~$T$ is a tensor field of type~$(r,s)$
with~$s>0$, then the~\emph{contraction} of~$T$ by~$X$ is the tensor
field~$i_XT$ of
type~$(r,s-1)$ obtained by plugging~$X$ into the first slot of~$T$.

\subsection{differential forms, wedge product}

Tensor fields of type~$(0,p)$ that are alternating are
called~\emph{differential forms} of degree~$p$, or simply~$p$-forms.
The set of~$p$-forms on~$M$ is denoted by~$\Omega^p(M)$.
Any tensor field of type~$(0,p)$ can be anti-symmetrized by summing over all
permutations of its inputs, weighted by the sign of each permutation.  Thus,
for~$p=2$, the tensor field~$(x,y)\mapsto\frac{T(x,y)-T(y,x)}2$ is
a~$2$-form.

If~$V$ is~$d$-dimensional, the set of anti-symmetric tensors of type~$(0,s)$ has
dimension~$\begin{pmatrix}d\\s\end{pmatrix}$.  Thus, if~$M$ is
a~$d$-dimensional manifold, the set of~$d$-forms~$\Omega^d(M)$ is pointwise
unidimensional.  Its elements are called~\emph{densities} on~$M$.
Similarly~$0$-forms are also uni-dimensional, and its elements are
called~\emph{potentials}.  The~$1$-forms are locally~$d$-dimensional and they
are called~\emph{gradients}.  Thus in dimension~$d>1$ we will always have two
types of ``scalar fields'' and two types of ``vector fields'':

\begin{tabular}{l|l|l|l}
	set & local dimension & meaning & units \\
	\hline
	$\Omega^0(M)=C^\infty(M)$ & 1 & scalar fields, potentials & $[]$ \\
	$\Omega^d(M)$ & 1 & densities & $[L^{-d}]$ \\
	$\Omega^1(M)$ & d & gradients & $[L^{-1}]$ \\
	$\mathcal{X}(M)$ & d& flows & $[L]$ \\
\end{tabular}

The~\emph{wedge product} is a map
\[
	\bigwedge:\Omega^p(M)\times\Omega^q(M)\to\Omega^{p+q}(M)
\]
defined by anti-symmetrizing the tensor product
\[
	\omega\wedge\eta=\frac{p+q}{p!q!}\mathrm{Alt}(\omega\otimes\eta).
\]

Since this definition is too abstract, let us see how it works out in~$\R^2$
and~$\R^3$.  Let us start with the canonical coordinate
functions~$x,y,z:\R^3\to\R$, etc.  They give rise to the canonical basis for
vector fields~$\ds\left\{
	\frac\partial{\partial x},
	\frac\partial{\partial y},
	\frac\partial{\partial z}
\right\}
$, which in turn can be used to define the canonical basis of~$1$-forms as
its dual base~$\left\{\ud x, \ud y, \ud z\right\}$.  By now this is just a
funky notation for the dual basis, but on the next section we will see
that~$\ud x$ is the~``$\ud$'' operator applied to the function~$x$.
Canonical bases of higher-order forms are obtained by taking wedge products
of~$1$-forms:

\begin{tabular}{l|l|l|l}
	set & dimension & basis & meaning \\
	\hline
	$\mathcal{X}(\R^2)$ & 2 &
	$\frac\partial{\partial x}, \frac\partial{\partial y}$ &
	flows, speeds \\
	$C^\infty(\R^2)=\Omega^0(\R^2)$ & 1 & $1$ & scalars, potentials\\
	$\Omega^1(\R^2)$ & 2 & $\ud x, \ud y$ & gradients, line elements\\
	$\Omega^2(\R^2)$ & 1 & $\ud x\wedge \ud y$ & densities, area elements\\
	\hline
	$\mathcal{X}(\R^3)$ & 3 &
	$\frac\partial{\partial x}, \frac\partial{\partial y},
	\frac\partial{\partial z}$ & flows, speeds\\
	$C^\infty(\R^3)=\Omega^0(\R^3)$ & 1 & $1$ & scalars, potentials\\
	$\Omega^1(\R^3)$ & 3 & $\ud x, \ud y, \ud z$ & gradients, line elements\\
	$\Omega^2(\R^3)$ & 3 &
	$\ud y\!\wedge\!\ud z, \ud z\!\wedge\!\ud x, \ud x\!\wedge\!\ud
	y$ & surface elements\\
	$\Omega^3(\R^3)$ & 1 & $\ud x\wedge\ud y\wedge\ud z$ & densities, volume
	elements\\
\end{tabular}



\subsection{exterior derivative, closed forms, betti numbers}

\subsection{chains, integrals and stokes theorem}

\subsection{currents and forms}

\subsection{lie derivatives (of functions, forms, tensors)}

\subsection{lie bracket}


\clearpage
\section{Riemannian geometry}

% GRAND SCHEME OF METRIC STUFF WITH METRIC-ONLY STUFF IN RED

\subsection{metric, length, energy}

\subsection{geodesic equations}

\subsection{musical isomorphisms, hodge duality}

\subsection{laplace beltrami}

\subsection{covariant derivative}

\subsection{parallel transport}

\subsection{killing vector fields (infinitessimal isometries)}

\subsection{hessian}

\subsection{curvatures}


\clearpage
\section{Other structured geometries}

\subsection{symplectic structure}

poisson bracket, volume form, cotangent bundle

\subsection{complex structure}

\subsection{kahlerian structure}


\clearpage
\section{Discrete case}



% vim:set tw=77 filetype=tex spell spelllang=en sw=2 ts=2:
