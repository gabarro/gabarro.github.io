

\newcommand{\1}{\mathbf{1}}
\newcommand{\R}{\mathbf{R}}
\newcommand{\T}{\mathbf{T}}
\newcommand{\Z}{\mathbf{Z}}
\newcommand{\ud}{\mathrm{d}}
\newcommand{\ds}{\displaystyle}

% \abs{x}         ->    |x|
% \Abs{x}         ->   ||x||
% \ABS{x}         ->  |||x|||
\newcommand{\abs}[1]{\left|#1\right|}
\newcommand{\Abs}[1]{\left\|#1\right\|}
\newcommand{\ABS}[1]{{\left\vert\kern-0.25ex\left\vert\kern-0.25ex\left\vert #1 \right\vert\kern-0.25ex\right\vert\kern-0.25ex\right\vert}}

% \parens{x}      ->  (x)
% \pairing{x}{y}  ->  <x,y>
\newcommand{\parens}[1]{\left(#1\right)} % (x)
\newcommand{\pairing}[2]{\left\langle #1,\,#2\right\rangle} % <x,y>




\title{summary of mechanics}

(just some notes while I read Arnold)

\section{Introduction}

There are three formalizations of classical mechanics.

In~{\bf Newtonian mechanics}, a physical system is described by a set of~$N$
particles moving in~$\R^3$.  The state of the system is given by the
positions and the speeds of all the
particles:~$(x,\dot x)\in\R^{3N}\times\R^{3N}$.  The evolution of the system
is governed by Newton's equation
\[
	\ddot x=F(\dot x, x,t)
\]
Where~$F:\R^{3N}\times\R^{3N}\times\R$ is a function determining the system
called the~\emph{force field}.

In~{\bf Lagrangian mechanics}, a physical system is described by a point in
a differential manifold~$M$ called the~\emph{configuration space}.  The tangent
space~$TM$ is called the~\emph{state space}.  The physics is governed by a
function~$L:TM\to\R$ called the~\emph{lagrangian}.  The trajectories of the system
are curves~$q:I\to M$ that minimize the functional
\[
	E(q)=\int_I L(q(t),\dot q(t),t)\mathrm{d}t.
\]
By the direct method in the calculus of variations, these trajectories
satisfy the Euler-Lagrange equations
\[
	\frac{\partial L}{\partial q}
	-\frac{\mathrm{d}}{\mathrm{d}t}\frac{\partial L}{\partial\dot q} = 0.
\]
Newtonian mechanics with a conservative force field~$(F=\nabla V)$
is a particular case of Lagrangian mechanics
mechanics when~$M=\R^{3N}$ and~$L=\frac12\left\|\dot q\right\|^2-V$.


In~{\bf Hamiltonian mechanics}, a physical system is described by a point in
a symplectic manifold~$\Omega$ called~\emph{phase space}.  The physics is
governed by a function~$H:\Omega\to\R$ called the~\emph{hamiltonian}.
Trajectories of the system are curves in~$\Omega$ satisfying Hamilton's
equations
\[
	\frac{\mathrm{d}q}{\mathrm{d}t}=\frac{\partial H}{\partial p},
	\qquad
	\frac{\mathrm{d}p}{\mathrm{d}t}=-\frac{\partial H}{\partial q}
\]
where~$(p,q)$ is any set of symplectic coordinates on~$\Omega$ (i.e., such
that the symplectic form is~$\omega=\mathrm{d}p_i\wedge\mathrm{d}q_i$).
Lagrangian mechanics is a particular case of Hamiltonian mechanics
when~$\Omega=T^*M$ and~$H$ is the legendre transform of~$L$.

On any hamiltonian system, the Hamilton-Jacobi equation
\[
	\frac{\partial S}{\partial t}
	=
	-H\left(q, \frac{\partial S}{\partial q}\right)
\]
is a single, first-order PDE that determines a scalar
function~$S:\Omega\to\R$ called the~\emph{action functional}.  The
function~$S$ gives a wave-like interpretation of whole set of possible
trajectories in the system: they are the integral lines of the field~$\nabla
S$.



{\bf why tho}\\
Each of the three formalisms above can be used for a complete description of
the physical world, so they are equivalent.  However, each flavor of
mechanics makes some things easier than others.

\section{Example: the harmonic oscillator}

Consider an object of mass~$m$ attached to a Hooke spring of constant~$k>0$.
We will model and study this physical system using all the formalisms
described above.

The {\bf Newtonian} formulation of this system is given by the differential
equation
\[
	m\ddot x = -kx
\]
where~$x$ is the distance of the mass to the position of equilibrium~$x=0$.

By the standard theory of ODE, given an initial position~$x=a$ and
speed~$\dot x = b$, the solution of this equation is determined uniquely to
be an harmonic oscillator of frequency~$\sqrt{\tfrac km}$:
\[
	x(t) =
	a\cos\left(\sqrt{\tfrac{k}{m}}\ t\right)
	+\frac{b}{\sqrt{\tfrac{k}{m}}}\sin\left(\sqrt{\tfrac{k}{m}} t\right)
\]
Newton's method shines here: we state an equation and we solve it completely.
The other formalisms will seem like overkill for this simple case.

In the {\bf Lagrangian} formulation of this problem we define a coordinate
system~$q$, with the same meaning as~$x$ above, and a Lagrangian function
\[
	L(q,\dot q) = \frac m2\dot q^2-\frac k2q^2
\]
trajectories of the system are thus minimizers of the energy functional
\[
	E(q) = \int L(q(t), \dot q(t))\ud t
\]
or equivalently, solutions of the corresponding Euler-Lagrange equations:
\[
	-kq - m\ddot q = 0
\]
whereby we recover the harmonic oscillator.
Notice that the lagrangian is~$L=T-V$ where~$T=\frac12m\dot q^2$ is the
kinetic energy and~$V=\frac12kq^2$ is the potential energy (such that the
force in the newtonian formulation is~$F=-\nabla V$).

The {\bf Hamiltonian} formulation takes place in~\emph{phase space} with
coordinates~$(q,p)\in\R^2$ where~$q=x$ is the position and~$p=m\dot x$ is the
momentum.  For the Hooke oscillator, the Hamiltonian function is
\[
	H = \frac{\tfrac1mp^2 + kq^2}2
\]
and the Hamilton equations are thus
\[
	\begin{cases}
	\dot q = \tfrac1mp & \\
	%\qquad
	\dot p = -kq
	\end{cases}
\]
Notice that this has a neat geometrical interpretation: the equations define
a vector field in phase space whose integral lines $t\mapsto(q(t),p(t))$ are
closed ellipses around the origin.  More explicitly, for an initial
condition~$q,p=a,b$, the solution is the following curve in phase space:
\[
	\begin{pmatrix}
		q(t) \\
		p(t)
	\end{pmatrix}
	\begin{pmatrix}
		q(t) \\
		p(t)
	\end{pmatrix}
\]

Finally, Hamilton-Jacobi equation
\[
	\frac{\partial S}{\partial t}
	=
	-H\left(q, \frac{\partial S}{\partial q}\right)
	=
	-\frac{kq^2}2
	-\frac 1{2m}\left(\frac{\partial S}{\partial q}\right)^2
\]
is a nonlinear first-order PDE in phase space.  The solution~$S$ is obviously
non-unique (we can, for example, add a constant to a solution to obtain a
different solution).



\section{Newtonian}

\subsection{Things that Newtonian mechanics does well}

\subsection{Things that Newtonian mechanics has trouble with}

\section{Lagrangian}

\subsection{Things that Lagrangian mechanics does well}

\subsection{Things that Lagrangian mechanics has trouble with}


\section{Hamiltonian}

\subsection{Things that Hamiltonian mechanics does well}

% vim:set tw=77 filetype=tex spell spelllang=en sw=2 ts=2:
