\title{Shapes with closed-form eigenmodes}

\newcommand{\R}{\mathbf{R}}
\newcommand{\Z}{\mathbf{Z}}
\newcommand{\Q}{\mathbf{Q}}
\newcommand{\C}{\mathbf{C}}
\newcommand{\U}{\mathbf{U}}
\newcommand{\T}{\mathbf{T}}
\newcommand{\ud}{\mathrm{d}}
\newcommand{\x}{\mathbf{x}}


This is a list of compact shapes whose Laplace-Beltrami
eigenfunctions can be expressed in terms of special functions.  It is
based mostly on the review articles by Grebenkov.

There are a just a few shapes where Helmholtz equation admits
closed-form solutions:

\begin{enumerate}
	\item The one-dimensional interval, using trigonometric functions.
	\item The disk, using Bessel functions
	\item The ellipse, using Mathieu functions
%	\item The surface of the sphere, using spherical harmonics (which
%		are polinomials of trigonometric functions)
	\item The equilateral triangle, using Lamé functions
\end{enumerate}

Other shapes can be also treated from these five ``basic'' types:

\begin{enumerate}
		\setcounter{enumi}{4}
	\item The rectangle in~$\R^d$, which is a separable product of
		intervals
	\item The cylinder, which is the product of a disk and an interval
	\item The right isosceles triangle, which is half a square and its
		eigenfunctions are obtained as those that vanish on the
		diagonal of the square
	\item The $90-60-30$ triangle, which is half an equilateral
		triangle
	\item Other separable products or symmetric subsets of treatable
		shapes
\end{enumerate}

Here we treat only the ``hard'' case of shapes with boundary.  The
case of compact manifolds without boundary
(circle, torus, sphere, ...) is somewhat easier.

\clearpage
\section{General considerations}

The basic result is the following.

\begin{proposition}
	Let~$\Omega$ be a compact manifold with boundary~$\partial\Omega$
	with Laplace-Beltrami operator~$\Delta_\Omega$.  Helmholtz
	equation with Dirichlet boundary conditions
	\begin{equation}\label{eq:helmholtz}
		\begin{cases}
			\Delta_\Omega u = -\lambda u & \Omega \\
			u = 0 & \partial\Omega
		\end{cases}
	\end{equation}
	has a numerable sequence of
	solutions~$\Delta_\Omega\varphi_n=-\lambda_n\varphi_n$,
	for~$n=1,2,3,\ldots$.  The numbers~$\{\lambda_n\}$, called
	the~\emph{spectrum} of~$\/\Omega$, are strictly positive and go to
	infinity.  The functions~$\varphi_n$, called the~\emph{normal
	modes}, are~$\mathcal{C}^{\infty}(\Omega)$ and form a Hilbert
	basis of~$L^2(\Omega)$.
\end{proposition}

This standard result (see for example Warner--Foundations of Differentiable
Manifolds an Lie Groups) settles the whole question of Helmholtz
equation, at least from the theoretical point of view.  In practice,
we may want to compute the spectrum or the normal modes numerically.
This can be done in general, by representing the shape~$\Omega$ by a
fine simplicial mesh, so that the equation in a continuous domain is
transformed as a linear algebra eigenvalue problem in finite
dimension.  Since the corresponding matrix is symmetric and positive
definite, the problem is particularly well posed.  But solving the
discretized problem exactly does not mean that we are close to the
real solution.  How fine must be the mesh so that the high-frequency
modes are correctly represented?  For that, it is very useful to have
a few simple domains where the solutions can be expressed explicitly
in terms of known functions.

The technique for solving equation~(\ref{eq:helmholtz}) in different
domains is always the same.  First, a family of solutions of~$\Delta u
= -\lambda u$ is determined.  This family depends on a few continuous
parameters.  It is then found that for some particular, discrete,
values of these parameters, the boundary condition is magically
satisfied.  These parameters are called the ``characteristic
values''.

The ideal case happens in the unit interval~$\Omega=[0,1]$.  The
equation without boundary conditions is~$u''=-\lambda u$, whose
solutions are of the general form~$\alpha\sin(\sqrt{\lambda}x)+
\beta\cos(\sqrt{\lambda}x)$.  By imposing the boundary conditions we
find that~$\beta=0$ and that~$\sqrt{\lambda}$ must be an integer
multiple of~$\pi$.

\clearpage
\section{Some ordinary differential equations}
We will solve Helmholtz equation on different domains by the method
of separation of variables, that turns a PDE into a collection of
independent ODE.  In this section we recall such ODE and their
solutions.

Recall that, in general, a linear second order ODE
\[
y'' = p(x)y'+q(x)y+r(x)
\]
will have two
families of independent solutions~$y(x) =
c_1y_{1}(x)+c_2y_{2}(x)$.
A first boundary condition like~$y(0)=0$  imposes a linear
relationship between~$c_1$ and~$c_2$.  A second boundary condition
like~$y(L)=0$ forces that~$L$ be a root of the solution, which may
exist only for specific values of parameters within $(p,q,r)$, and these
roots can be tabulated.

Let us see some examples.

\subsection{Trigonometric functions}

The ``trigonometric ODE''
\begin{equation}\label{eq:trigo}
	y'' = -\lambda y
\end{equation}
for~$\lambda>0$
has general solutions~$y(x)=c_1\sin\left(\sqrt\lambda x\right)
+ c_2\cos\left(\sqrt\lambda x\right)$.  The functions~$\sin$
and~$\cos$ are well known and can be evaluated on a computer by
standard math libraries like \verb+math.h+, etc.
Moreover, the zeros of the functions~$\sin$ and~$\cos$ are tabulated: they are
integer multiples of an universal constant~$\pi$, which is also well
known and available everywhere.

The condition~$y(0)=0$ implies that~$c_2=0$ and does not impose any
constraint on~$\lambda$.  Now, the
condition~$y(L)=0$ says~$\sin\left(\sqrt\lambda L\right)=0$
thus~$\sqrt\lambda L$ must be an integer multiple of~$\pi$.  This
implies that the ODE will only have a solution satisfying this
boundary condition for a discrete set of
values~$\lambda_k=\frac{\pi^2 k^2}{L^2}$, $k=1,2,3,\ldots$.

\includegraphics{odesincos.png}
%SCRIPT plambda zero:1x1000 ":j :h / 7 * dup pi * 7 / sin join" -o tmp/sincos1.txt
%SCRIPT plambda zero:1x1000 ":j :h / 7 * dup pi * 7 / 2 * sin join" -o tmp/sincos2.txt
%SCRIPT plambda zero:1x1000 ":j :h / 7 * dup pi * 7 / 3 * sin join" -o tmp/sincos3.txt
%SCRIPT gnuplot <<END >odesincos.png
%SCRIPT set term pngcairo lw 1.5
%SCRIPT set title "Dirichlet Sines"
%SCRIPT set samples 1000
%SCRIPT set key left bottom
%SCRIPT plot [-1:8] [-1.5:1.5] 0,\
%SCRIPT  "tmp/sincos1.txt" w lines title "sin(πx/7)",\
%SCRIPT  "tmp/sincos2.txt" w lines title "sin(2πx/7)",\
%SCRIPT  "tmp/sincos3.txt" w lines title "sin(3πx/7)"
%SCRIPT END

\subsection{Bessel functions}

The Bessel equation
\begin{equation}\label{eq:bessel}
	x^2y''+xy'+(x^2-\alpha^2)y = 0
\end{equation}
has general solutions~$y(x)=c_1 J_{\alpha}(x)+c_2 Y_{\alpha}(x)$.
The functions~$J_\alpha$ and ~$Y_\alpha$ are well known and can be
evaluated on a computer by standard math libraries
like~\verb+math.h+, etc.
They are called the Bessel functions of the first and second kind
respectively.  The zeros of these functions are also well-known, and
while they are not tabulated inside the most basic libraries
like~\verb+math.h+, they are easily available.  For example,
in~C, you can use the function~\verb+gsl_sf_bessel_zero_Jnu(a,k)+ from the
GNU Scientific Library, to get the~$k$-th zero of~$J_a$.  In Python,
you can use the GSL Python bindings, or~\verb+scipy.special.jn_zeros+
and related functions.

Since the functions~$Y_\alpha$ have a vertical asymptote at~$x=0$,
the condition that~$y(0)$ is finite implies that~$c_2=0$.  Now, the
condition~$y(L)=0$ says~$J_\alpha(L)=0$.  The positive roots of~$J_\alpha$ are
tabulated, so that for each~$\alpha\ge 0$ there is an increasing
sequence~$j_{\alpha,k}$ giving all the roots of~$J_\alpha$.
By scaling each~$J_\alpha$ by its successive roots, we obtain a
family of functions with~$1,2,3,\ldots$ oscillations inside~$[0,L]$.

\includegraphics{odebessel.png}
%SCRIPT plambda zero:1x1000 ":j :h / 20 * dup 0 rot bessel-Jn join" -o tmp/bess0.txt
%SCRIPT plambda zero:1x1000 ":j :h / 20 * dup 1 rot bessel-Jn join" -o tmp/bess1.txt
%SCRIPT plambda zero:1x1000 ":j :h / 20 * dup 2 rot bessel-Jn join" -o tmp/bess2.txt
%SCRIPT plambda zero:1x1000 ":j :h / 20 * dup 3 rot bessel-Jn join" -o tmp/bess3.txt
%SCRIPT plambda zero:1x1000 ":j :h / 20 * dup 4 rot bessel-Jn join" -o tmp/bess4.txt
%SCRIPT plambda zero:1x1000 ":j :h / 20 * dup 13 rot bessel-Jn join" -o tmp/bess13.txt
%SCRIPT gnuplot <<END >odebessel.png
%SCRIPT set term pngcairo lw 1.5
%SCRIPT set title "Bessel functions of the first kind\n(raw, not scaled)"
%SCRIPT set key left bottom
%SCRIPT plot [-1:21] [-1.5:1.5] 0,\
%SCRIPT  "tmp/bess0.txt" w lines title "J_0(x)",\
%SCRIPT  "tmp/bess1.txt" w lines title "J_1(x)",\
%SCRIPT  "tmp/bess2.txt" w lines title "J_2(x)",\
%SCRIPT  "tmp/bess3.txt" w lines title "J_3(x)",\
%SCRIPT  "tmp/bess4.txt" w lines title "J_4(x)",\
%SCRIPT  "tmp/bess13.txt" w lines title "J_{13}(x)"
%SCRIPT END

\includegraphics{besselnorm0.png}
%SCRIPT plambda zero:1x1000 ":j :h / 7 * dup 7 / 0 1 bessel-zJn * 0 rot bessel-Jn join" -o tmp/bn01.txt
%SCRIPT plambda zero:1x1000 ":j :h / 7 * dup 7 / 0 2 bessel-zJn * 0 rot bessel-Jn join" -o tmp/bn02.txt
%SCRIPT plambda zero:1x1000 ":j :h / 7 * dup 7 / 0 3 bessel-zJn * 0 rot bessel-Jn join" -o tmp/bn03.txt
%SCRIPT plambda zero:1x1000 ":j :h / 7 * dup 7 / 0 4 bessel-zJn * 0 rot bessel-Jn join" -o tmp/bn04.txt
%SCRIPT plambda zero:1x1000 ":j :h / 7 * dup 7 / 0 5 bessel-zJn * 0 rot bessel-Jn join" -o tmp/bn05.txt
%SCRIPT gnuplot <<END >besselnorm0.png
%SCRIPT set term pngcairo lw 1.5
%SCRIPT set title "Bessel functions of the first kind\n(successive scalings of J_0)"
%SCRIPT set key left bottom
%SCRIPT plot [-1:8] [-1.5:1.5] 0,\
%SCRIPT  "tmp/bn01.txt" w lines title "J_0( j_{0,1}x/7 )",\
%SCRIPT  "tmp/bn02.txt" w lines title "J_0( j_{0,2}x/7 )",\
%SCRIPT  "tmp/bn03.txt" w lines title "J_0( j_{0,2}x/7 )",\
%SCRIPT  "tmp/bn04.txt" w lines title "J_0( j_{0,2}x/7 )",\
%SCRIPT  "tmp/bn05.txt" w lines title "J_0( j_{0,5}x/7 )"
%SCRIPT END

\includegraphics{besselnorm1.png}
%SCRIPT plambda zero:1x1000 ":j :h / 7 * dup 7 / 1 1 bessel-zJn * 1 rot bessel-Jn join" -o tmp/bn11.txt
%SCRIPT plambda zero:1x1000 ":j :h / 7 * dup 7 / 1 2 bessel-zJn * 1 rot bessel-Jn join" -o tmp/bn12.txt
%SCRIPT plambda zero:1x1000 ":j :h / 7 * dup 7 / 1 3 bessel-zJn * 1 rot bessel-Jn join" -o tmp/bn13.txt
%SCRIPT plambda zero:1x1000 ":j :h / 7 * dup 7 / 1 4 bessel-zJn * 1 rot bessel-Jn join" -o tmp/bn14.txt
%SCRIPT plambda zero:1x1000 ":j :h / 7 * dup 7 / 1 5 bessel-zJn * 1 rot bessel-Jn join" -o tmp/bn15.txt
%SCRIPT gnuplot <<END >besselnorm1.png
%SCRIPT set term pngcairo lw 1.5
%SCRIPT set title "Bessel functions of the first kind\n(successive scalings of J_1)"
%SCRIPT set key left bottom
%SCRIPT plot [-1:8] [-1.5:1.5] 0,\
%SCRIPT  "tmp/bn11.txt" w lines title "J_1( j_{1,1}x/7 )",\
%SCRIPT  "tmp/bn12.txt" w lines title "J_1( j_{1,2}x/7 )",\
%SCRIPT  "tmp/bn13.txt" w lines title "J_1( j_{1,2}x/7 )",\
%SCRIPT  "tmp/bn14.txt" w lines title "J_1( j_{1,2}x/7 )",\
%SCRIPT  "tmp/bn15.txt" w lines title "J_1( j_{1,5}x/7 )"
%SCRIPT END

\includegraphics{besselnorm2.png}
%SCRIPT plambda zero:1x1000 ":j :h / 7 * dup 7 / 2 1 bessel-zJn * 2 rot bessel-Jn join" -o tmp/bn21.txt
%SCRIPT plambda zero:1x1000 ":j :h / 7 * dup 7 / 2 2 bessel-zJn * 2 rot bessel-Jn join" -o tmp/bn22.txt
%SCRIPT plambda zero:1x1000 ":j :h / 7 * dup 7 / 2 3 bessel-zJn * 2 rot bessel-Jn join" -o tmp/bn23.txt
%SCRIPT plambda zero:1x1000 ":j :h / 7 * dup 7 / 2 4 bessel-zJn * 2 rot bessel-Jn join" -o tmp/bn24.txt
%SCRIPT plambda zero:1x1000 ":j :h / 7 * dup 7 / 2 5 bessel-zJn * 2 rot bessel-Jn join" -o tmp/bn25.txt
%SCRIPT gnuplot <<END >besselnorm2.png
%SCRIPT set term pngcairo lw 1.5
%SCRIPT set title "Bessel functions of the first kind\n(successive scalings of J_2)"
%SCRIPT set key left bottom
%SCRIPT plot [-1:8] [-1.5:1.5] 0,\
%SCRIPT  "tmp/bn21.txt" w lines title "J_2( j_{2,1}x/7 )",\
%SCRIPT  "tmp/bn22.txt" w lines title "J_2( j_{2,2}x/7 )",\
%SCRIPT  "tmp/bn23.txt" w lines title "J_2( j_{2,2}x/7 )",\
%SCRIPT  "tmp/bn24.txt" w lines title "J_2( j_{2,2}x/7 )",\
%SCRIPT  "tmp/bn25.txt" w lines title "J_2( j_{2,5}x/7 )"
%SCRIPT END

\includegraphics{besselnorm13.png}
%SCRIPT plambda zero:1x1000 ":j :h / 7 * dup 7 / 13 1 bessel-zJn * 13 rot bessel-Jn join" -o tmp/bn131.txt
%SCRIPT plambda zero:1x1000 ":j :h / 7 * dup 7 / 13 2 bessel-zJn * 13 rot bessel-Jn join" -o tmp/bn132.txt
%SCRIPT plambda zero:1x1000 ":j :h / 7 * dup 7 / 13 3 bessel-zJn * 13 rot bessel-Jn join" -o tmp/bn133.txt
%SCRIPT plambda zero:1x1000 ":j :h / 7 * dup 7 / 13 4 bessel-zJn * 13 rot bessel-Jn join" -o tmp/bn134.txt
%SCRIPT plambda zero:1x1000 ":j :h / 7 * dup 7 / 13 5 bessel-zJn * 13 rot bessel-Jn join" -o tmp/bn135.txt
%SCRIPT gnuplot <<END >besselnorm13.png
%SCRIPT set term pngcairo lw 1.5
%SCRIPT set title "Bessel functions of the first kind\n(successive scalings of J_{13})"
%SCRIPT set key left bottom
%SCRIPT plot [-1:8] [-1.5:1.5] 0,\
%SCRIPT  "tmp/bn131.txt" w lines title "J_{13}( j_{13,1}x/7 )",\
%SCRIPT  "tmp/bn132.txt" w lines title "J_{13}( j_{13,2}x/7 )",\
%SCRIPT  "tmp/bn133.txt" w lines title "J_{13}( j_{13,2}x/7 )",\
%SCRIPT  "tmp/bn134.txt" w lines title "J_{13}( j_{13,2}x/7 )",\
%SCRIPT  "tmp/bn135.txt" w lines title "J_{13}( j_{13,5}x/7 )"
%SCRIPT END

\includegraphics{besselnormr1.png}
%SCRIPT plambda zero:1x1000 ":j :h / 7 * dup 7 / 0 1 bessel-zJn * 0 rot bessel-Jn join" -o tmp/bnr01.txt
%SCRIPT plambda zero:1x1000 ":j :h / 7 * dup 7 / 1 1 bessel-zJn * 1 rot bessel-Jn join" -o tmp/bnr11.txt
%SCRIPT plambda zero:1x1000 ":j :h / 7 * dup 7 / 2 1 bessel-zJn * 2 rot bessel-Jn join" -o tmp/bnr21.txt
%SCRIPT plambda zero:1x1000 ":j :h / 7 * dup 7 / 3 1 bessel-zJn * 3 rot bessel-Jn join" -o tmp/bnr31.txt
%SCRIPT plambda zero:1x1000 ":j :h / 7 * dup 7 / 4 1 bessel-zJn * 4 rot bessel-Jn join" -o tmp/bnr41.txt
%SCRIPT plambda zero:1x1000 ":j :h / 7 * dup 7 / 13 1 bessel-zJn * 13 rot bessel-Jn join" -o tmp/bnr131.txt
%SCRIPT gnuplot <<END >besselnormr1.png
%SCRIPT set term pngcairo lw 1.5
%SCRIPT set title "Bessel functions of the first kind\n(scalings of a few J_n at their first root)"
%SCRIPT set key left bottom
%SCRIPT plot [-1:8] [-1.5:1.5] 0,\
%SCRIPT  "tmp/bnr01.txt" w lines title "J_0( j_{0,1}x/7 )",\
%SCRIPT  "tmp/bnr11.txt" w lines title "J_1( j_{1,1}x/7 )",\
%SCRIPT  "tmp/bnr21.txt" w lines title "J_2( j_{2,1}x/7 )",\
%SCRIPT  "tmp/bnr31.txt" w lines title "J_3( j_{3,1}x/7 )",\
%SCRIPT  "tmp/bnr41.txt" w lines title "J_4( j_{4,1}x/7 )",\
%SCRIPT  "tmp/bnr131.txt" w lines title "J_{13}( j_{13,1}x/7 )"
%SCRIPT END

\includegraphics{besselnormr2.png}
%SCRIPT plambda zero:1x1000 ":j :h / 7 * dup 7 / 0 2 bessel-zJn * 0 rot bessel-Jn join" -o tmp/bnr02.txt
%SCRIPT plambda zero:1x1000 ":j :h / 7 * dup 7 / 1 2 bessel-zJn * 1 rot bessel-Jn join" -o tmp/bnr12.txt
%SCRIPT plambda zero:1x1000 ":j :h / 7 * dup 7 / 2 2 bessel-zJn * 2 rot bessel-Jn join" -o tmp/bnr22.txt
%SCRIPT plambda zero:1x1000 ":j :h / 7 * dup 7 / 3 2 bessel-zJn * 3 rot bessel-Jn join" -o tmp/bnr32.txt
%SCRIPT plambda zero:1x1000 ":j :h / 7 * dup 7 / 4 2 bessel-zJn * 4 rot bessel-Jn join" -o tmp/bnr42.txt
%SCRIPT plambda zero:1x1000 ":j :h / 7 * dup 7 / 13 2 bessel-zJn * 13 rot bessel-Jn join" -o tmp/bnr132.txt
%SCRIPT gnuplot <<END >besselnormr2.png
%SCRIPT set term pngcairo lw 1.5
%SCRIPT set title "Bessel functions of the first kind\n(scalings of a few J_n at their second root)"
%SCRIPT set key left bottom
%SCRIPT plot [-1:8] [-1.5:1.5] 0,\
%SCRIPT  "tmp/bnr02.txt" w lines title "J_0( j_{0,2}x/7 )",\
%SCRIPT  "tmp/bnr12.txt" w lines title "J_1( j_{1,2}x/7 )",\
%SCRIPT  "tmp/bnr22.txt" w lines title "J_2( j_{2,2}x/7 )",\
%SCRIPT  "tmp/bnr32.txt" w lines title "J_3( j_{3,2}x/7 )",\
%SCRIPT  "tmp/bnr42.txt" w lines title "J_4( j_{4,2}x/7 )",\
%SCRIPT  "tmp/bnr132.txt" w lines title "J_{13}( j_{13,2}x/7 )"
%SCRIPT END

\includegraphics{besselnormr13.png}
%SCRIPT plambda zero:1x1000 ":j :h / 7 * dup 7 / 0 13 bessel-zJn * 0 rot bessel-Jn join" -o tmp/bnr013.txt
%SCRIPT plambda zero:1x1000 ":j :h / 7 * dup 7 / 1 13 bessel-zJn * 1 rot bessel-Jn join" -o tmp/bnr113.txt
%SCRIPT plambda zero:1x1000 ":j :h / 7 * dup 7 / 2 13 bessel-zJn * 2 rot bessel-Jn join" -o tmp/bnr213.txt
%SCRIPT plambda zero:1x1000 ":j :h / 7 * dup 7 / 3 13 bessel-zJn * 3 rot bessel-Jn join" -o tmp/bnr313.txt
%SCRIPT plambda zero:1x1000 ":j :h / 7 * dup 7 / 4 13 bessel-zJn * 4 rot bessel-Jn join" -o tmp/bnr413.txt
%SCRIPT plambda zero:1x1000 ":j :h / 7 * dup 7 / 13 13 bessel-zJn * 13 rot bessel-Jn join" -o tmp/bnr1313.txt
%SCRIPT gnuplot <<END >besselnormr13.png
%SCRIPT set term pngcairo lw 1.5
%SCRIPT set title "Bessel functions of the first kind\n(scalings of a few J_n at their thirteenth root)"
%SCRIPT set key left bottom
%SCRIPT plot [-1:8] [-1.5:1.5] 0,\
%SCRIPT  "tmp/bnr013.txt" w lines title "J_0( j_{0,13}x/7 )",\
%SCRIPT  "tmp/bnr113.txt" w lines title "J_1( j_{1,13}x/7 )",\
%SCRIPT  "tmp/bnr213.txt" w lines title "J_2( j_{2,13}x/7 )",\
%SCRIPT  "tmp/bnr313.txt" w lines title "J_3( j_{3,13}x/7 )",\
%SCRIPT  "tmp/bnr413.txt" w lines title "J_4( j_{4,13}x/7 )",\
%SCRIPT  "tmp/bnr1313.txt" w lines title "J_{13}( j_{13,13}x/7 )"
%SCRIPT END


\subsection{Mathieu functions}

Let~$a$ and~$q$ be real-valued parameters.
The Mathieu equation
\begin{equation}\label{eq:mathieu}
	y'' + (a-2q\cos(2x))y = 0
\end{equation}
and the modified Mathieu equation
\begin{equation}\label{eq:mathieumod}
	y'' - (a-2q\cosh(2x))y = 0
\end{equation}
are second order linear equations (with non-constant coefficients)
whose solutions exist by the standard theory and are called Mathieu
functions (and modified Mathieu functions).

Let us start with the solutions of equation~(\ref{eq:matieu}).  For
general values of~$a$ and $q$, the solutions are not periodic.  But,
since the coefficients are periodic, it turns out that there are
periodic solutions, for discrete values of~$a$ and~$q$.  Typically,
the parameter~$q$ is left free, and then there are choices
of~$a$ called the ``characteristic values'' for that~$q$, so that the
solutions are~$2\pi$-periodic.  The characteristic values are
denoted~$a_n(q)$ and~$b_n(q)$ for~$n=1,2,3$, and the corresponding
solutions are the Mathieu functions~$ce_n(q,x)$ and~$se_n(q,x)$.

\includegraphics{odemath0.png}
%SCRIPT plambda zero:1x1000 ":j :h / 20 * dup 0 rot bessel-Jn join" -o tmp/bess0.txt
%SCRIPT plambda zero:1x1000 ":j :h / 20 * dup 1 rot bessel-Jn join" -o tmp/bess1.txt
%SCRIPT plambda zero:1x1000 ":j :h / 20 * dup 2 rot bessel-Jn join" -o tmp/bess2.txt
%SCRIPT plambda zero:1x1000 ":j :h / 20 * dup 3 rot bessel-Jn join" -o tmp/bess3.txt
%SCRIPT plambda zero:1x1000 ":j :h / 20 * dup 4 rot bessel-Jn join" -o tmp/bess4.txt
%SCRIPT plambda zero:1x1000 ":j :h / 20 * dup 13 rot bessel-Jn join" -o tmp/bess13.txt
%SCRIPT gnuplot <<END >odemath0.png
%SCRIPT set term pngcairo lw 1.5
%SCRIPT set title "Bessel functions of the first kind"
%SCRIPT set key left bottom
%SCRIPT plot [-1:21] [-1.5:1.5] 0,\
%SCRIPT  "tmp/bess0.txt" w lines title "J_0(x)",\
%SCRIPT  "tmp/bess1.txt" w lines title "J_1(x)",\
%SCRIPT  "tmp/bess2.txt" w lines title "J_2(x)",\
%SCRIPT  "tmp/bess3.txt" w lines title "J_3(x)",\
%SCRIPT  "tmp/bess4.txt" w lines title "J_4(x)",\
%SCRIPT  "tmp/bess13.txt" w lines title "J_{13}(x)"
%SCRIPT END






\clearpage
\section{The interval}

Let~$\Omega=[0,1]$.  The equation without boundary conditions is
\begin{equation}\label{eq:sincos}
	u''=-\lambda u
\end{equation}
solutions of this equation for an arbitrary~$\lambda\in\C$ are of the form
\[
	u(x)=
	\alpha_1 e^{\mu x}
	+
	\alpha_2 e^{-\mu x}
	+
	\alpha_3 e^{i\mu x}
	+
	\alpha_4 e^{-i\mu x}
\]
where~$\mu^2=-\lambda$.  For~$\lambda>0$ and real-valued~$u$, this
gives~$\alpha_1=\alpha_2=0$.  Thus  we are interested in solutions
of~(\ref{eq:sincos}) of the form
\[
	u(x)=\alpha\sin\left(\sqrt{\lambda}
	x\right)+\beta\cos\left(\sqrt{\lambda}x\right).
\]
Now, the boundary conditions are~$u(0)=0$ and~$u(1)=0$.  The first
boundary condition implies that~$\beta=0$.  The second boundary
condition now says
\begin{equation}\label{eq:sinroots}
	\sin\left(\sqrt{\lambda}\right) = 0
\end{equation}
Since the sine function has roots at integer multiples of~$\pi$, the
solutions of this equation are the numbers~$\lambda_n=n^2\pi^2$.

The computation above proves that, for~$\Omega=[0,1]$, the
spectrum and its corresponding normal modes are
\[
	\lambda_n = \pi^2n^2
	\qquad n=1,2,3,\ldots
\]
\[
	\varphi_n(x) = \sin\left(\pi n x\right)
	\qquad n=1,2,3,\ldots
\]
On an interval of arbitrary length~$\Omega=[0,L]$ the result is
obtained by a linear change of variable
\[
	\varphi_n(x) = \sin\frac{\pi n x}L
	\qquad n=1,2,3,\ldots
\]
\[
	\lambda_n = \frac{\pi^2n^2}{L^2}
	\qquad n=1,2,3,\ldots
\]


\clearpage
\section{The rectangle}

Consider now~$\Omega=[0,A]\times[0,B]\subseteq\R^2$.
The Laplacian in two variables is separable: if~$u(x,y)=p(x)q(y)$
then~$\Delta u = p''q + q''p$.
Helmholtz equation now says
\[
	p''(x)q(y)+p(x)q''(y)=-\lambda p(x)q(y)
\]
or, equivalently,
\[
	\left(p''(x)+\lambda p(x)\right)q(y) = -p(x)q''(y)
\]
This equation will hold whenever the following two do, for the same
``characteristic constant''~$\mu$:
\[
	\begin{cases}
		q'' = -\mu q & \\
		p'' = (\mu-\lambda) p
	\end{cases}
\]
which, together with their respective boundary conditions, have the solutions
\[
	p_m(x) = \sin\frac{\pi n x}{A} \qquad m=1,2,3,\ldots
\]
\[
	q_n(y) = \sin\frac{\pi n y}{B} \qquad n=1,2,3,\ldots
\]
%Thus we can obtain solutions of
%the two-dimensional Helmholtz equation by solving separately
%the two equations
%\[
%	p''=-\lambda p
%	\qquad\qquad
%	q''=-\lambda q
%\]
%and then building~$u=pq$.  Notice that this will produce~\emph{some}
%solutions of Helmholtz equation, but of course not all of them (in
%particular, the property of being separable is not even closed by
%linear combinations!).  Some kind of completeness of these solutions
%will be needed to to proven later, after imposing the boundary
%conditions.

%In the case of the rectangle, the boundary conditions can be imposed
%separately for~$p$ and~$q$.
Thus the family of functions
\[
	\varphi_{m,n}(x,y)=\sin\frac{\pi m x}A\sin\frac{\pi n y}B
\]
satisfy the boundary conditions and satisfy Helmoltz equation with
eigenvalues
\[
	\lambda_{m,n} =
	\frac{\pi^2m^2}{A^2}
	+
	\frac{\pi^2n^2}{B^2}.
\]
for~$m,n=1,2,3,\ldots$.
The case of the rectangle is particularly simple, but it has a
similar
structure than the cases for the disk and the ellipse.
The difference will be that the condition of separability is
different, but the structure of the reasoning is the same.



\clearpage
\section{The disk}

Consider the unit disk~$\Omega=\left\{\x\in\R^2\ :\
\|\x\|\le1\right\}$.  This set is not separable in Cartesian
coordinates~$\x=(x,y)$, but it is separable in polar
coordinates~$(r,\theta)$, defined by
\[
	\begin{cases}
		x = r\cos\theta \\
		y = r\sin\theta
	\end{cases}
	\qquad\qquad
	\qquad
	\begin{cases}
		r = \sqrt{x^2+y^2} \\
		\theta = \mathrm{atan2}(y,x)
	\end{cases}
\]
to write Helmholtz equation in polar coordinates, we need to
express the Laplacian in terms of derivatives of~$r$ and~$\theta$:
\[
	\Delta
	\ = \ %
	\frac{\partial^2}{\partial x^2}
	+
	\frac{\partial^2}{\partial y^2}
	\ = \ %
	\frac{\partial^2}{\partial r^2}
	+
	\frac 1r
	\frac{\partial}{\partial r}
	+
	\frac 1{r^2}
	\frac{\partial^2}{\partial \theta^2}.
\]
Now we can apply separation of variables:
if~$u(r,\theta)=f(r)g(\theta)$ then Helmholtz equation becomes
\[%\Delta u =
f''(r)g(\theta)+\frac{f'(r)}rg(\theta)+\frac{f(r)}{r^2}g''(\theta)
=-\lambda f(r)g(\theta)
\]
or, rearranging the terms,
\[
	\left(
f''(r)
+
\frac{f'(r)}r
+\lambda f(r)
\right)g(\theta)
=
-
\frac{f(r)}{r^2}g''(\theta)
\]
This equation will hold whenever the following two do, for the same
``characteristic constant''~$\mu$:
\[
	\begin{cases}
		g''=-\mu g \\
		f''+{f'}/{r}+\lambda f = \mu{f}/{r^2}
	\end{cases}
\]
These are now two separate ordinary equations on the
variables~$\theta$ and~$r$, respectively.  Let us consider them
separately.

The first equation~$g''=-\mu g$ is an old friend.  Its solutions are
sines and cosines of~$\sqrt{\mu}\theta$.  Now, if we want that this
represents something in polar coordinates, then~$g$ must be~$2\pi$
periodic, so~$\sqrt{\mu}$ must be an integer multiple of~$2\pi$,
thus~$\mu_k=k^2$ and we have two families of solutions
for~$g$:
\[
	g^0_k(\theta) = \cos(k \theta)
	\quad k=0,1,2,\ldots
\]
\[
	g^1_k(\theta) = \sin(k \theta)
	\quad k=1,2,3,\ldots
\]

Now consider the second equation
\[
	r^2f''(r) + rf'(r)
	+\left(\lambda r^2
		-
	%4\pi^2
	k^2
	\right)f(r) = 0
\]
and apply the change of variable~$x=\sqrt\lambda
r$,~$y(x)=f\left(x/\sqrt\lambda\right)$ to rewrite it as
\[
		x^2y''+xy'+\left(x^2-
		%4\pi^2
	k^2\right)y=0
\]
This is Bessel equation, and its solutions bounded around $x=0$ are
the Bessel functions of the first kind with integer order~$J_k(x)$.

\section{The ellipse}


\subsection{Digression on elliptic coordinates}

Fix~$c>0$.  Elliptic coordinates in~$\R^2$ with foci at~$\pm c$ are
defined by
\begin{equation}\label{eq:elliptic}
	\begin{cases}
		x = c\ \cosh\rho\cos\theta \\
		y = c\ \sinh\rho\sin\theta
	\end{cases}
\end{equation}
where~$\rho\in[0,+\infty)$ and~$\theta\in[0,2\pi)$.
They could be called hyperbolic coordinates as well.
Curves of constant~$\rho$ are ellipses, and curves of
constant~$\theta$ are hyperbolas, always with the same foci at~$\pm
c$.  We can rewrite equation~(\ref{eq:elliptic}) using complex
numbers as~$z=c\ \cosh\zeta$,  where~$z=x+iy$ and~$\zeta=\rho+i\theta$.

I find this parametrization difficult to understand,
because when~$c$ goes to~$0$ it should approach classical polar
coordinates, but everything seems to go to zero.
The problem is that the parameter~$\rho$ is not the radius, but the
logarithm of the radius!  More precisely, by the change of
variable~$r=\frac12ce^\rho$ equation~(\ref{eq:elliptic}) becomes
\begin{equation}\label{eq:ellipticr}
	\begin{cases}
		x = \left(r + \frac{c^2}{4r}\right)\cos\theta \\
		y = \left(r - \frac{c^2}{4r}\right)\sin\theta
	\end{cases}
\end{equation}
This is equivalent, but now it is clear that when~$c=0$ we recover
polar coordinates, and that for~$c>0$ the vertical axis is shorter
than the horizontal one.  The major and minor axis of the ellipse are
visibly~$2r\pm c^2/2r$.
Notice however that the
parametrization~(\ref{eq:ellipticr}) needs to consider
only~$r\in[c/2,+\infty)$.  Values of~$r\in(0,c/2)$ correspond
to~$\rho<0$, and traverse the other side of the half-plane.


In practice, we will need the inverse transformation
of~(\ref{eq:elliptic}).  That is, given a point~$(x,y)$, how to
find its elliptic coordinates?  This entails solving a second-degree
equation.  More precisely, we use the following algorithm:
\begin{align*}
	b   & \leftarrow x^2+y^2-c^2 \\
	p,q & \leftarrow \left(-b\pm\sqrt{b^2+4c^2y^2}\right)/{2c^2} \\
	\theta & \leftarrow \arcsin\sqrt{p} \\
	\rho & \leftarrow  \frac12\log\left(1-2q+2\sqrt{q^2-q}\right)
\end{align*}
Notice that, by construction, we have~$p\in[0,1]$ and~$q\in[-1,0]$.
Since the arcsine
always gives a~$\theta$ in the first quadrant, to finish the
algorithm we must move~$\theta$ to the appropriate quadrant by
looking at the signs of~$x$ and~$y$.

TODO: find a ``reparametrized algorithm'' using atan2 instead of
arcsin that directly finds~$\theta$ in the good quadrant.  Maybe this
will help to interpret the inverse transform when~$c\to0$, which is
ugly right now.  Notice that~$\arcsin\sqrt
p=\arctan\sqrt{\frac{p}{1-p}}$, but this cannot lead to the correct
solution, since the signs of~$x$ and~$y$ are lost once we square
them to form~$b,p,q$.  It must be an expression directly in terms
of~$x,y$.  SOLUTION: first compute~$\rho$ and then
set~$\theta=\mathrm{atan2}\left(y,\tanh(\rho) x\right)$.

\subsection{Elliptic separation of variables}

The Laplacian in elliptic coordinates
is~$\displaystyle\Delta=
\frac1{c^2\left(\sinh^2\rho+\sin^2\theta\right)}
\left(
	\frac{\partial^2}{\partial\rho^2}
	+
	\frac{\partial^2}{\partial\theta^2}
\right)
$.
To study Helmholtz equation on an elliptic domain it is natural to
consider functions that are separable in elliptic
coordinates~$u(\rho,\theta)=f(\rho)g(\theta)$.  The condition~$\Delta
u=-\lambda u$ thus reads
\[
\frac1{c^2\left(\sinh^2\rho+\sin^2\theta\right)}
\left(f''g+fg''\right)
=\lambda fg
\]
which, using the
identity~$\sinh^2\rho+\sin^2\theta=\frac{\cosh{2\rho}-\cos{2\theta}}2$
and
rearranging, becomes
\[
	\frac{\lambda c^2}2\cosh(2\rho)
	+
	\frac{f''(\rho)}{f(\rho)}
	\quad = \quad
	\frac{\lambda c^2}2\cos(2\theta)
	-\frac{g''(\theta)}{g(\theta)}.
\]
Now,
since each side of this equality is a function of a different
variables, they both be equal to a constant, that we will denote~$a$.
This leads to the following ODE for~$f$ and~$g$:
\begin{equation}\label{eq:mathieuell}
	g''(\theta)+\left(a-2q\cos2\theta\right)g(\theta)=0
	\qquad
	f''(\rho)-\left(a-2q\cosh2\rho\right)f(\rho)=0
\end{equation}
where~$q=\lambda c^2/4$.  Here we recognize the Mathieu
equations, whose solutions are the Mathieu functions.  Here we will
be interested in the Mathieu functions of the first kind, which are
the periodic solutions of~$g$; and modified Mathieu functions of the
first kind, which are the two families of solutions of~$f$.



\section{The equilateral triangle}

% vim:set ts=3 sw=3 tw=69 filetype=tex spell spelllang=en:
