\title{All of math}

%^ C1V
\paragraph{Calculus of One Variable.}
Definition of real numbers, limit of a sequence, sum of a series, limit of a
function, continuous function, theorem of Bolzano, derivative, mean value
theorem, Taylor theorem, Riemann integral, fundamental theorem of calculus.
Methods of computation of primitives.
[Spivak's Calculus; Arnold's Analysis (the small)]
{}

%^ LA
\paragraph{Linear Algebra.}
Vector spaces.  Basis of a vector space.  Linear maps between spaces.  Matrix
of a linear map.  Linear endomorphism.  Determinant.  Cayley-Hamilton
theorem.  Factorization of maps/matrices.  Jordan theorem.  Eigensystems.
[Strang]
{}

%^ BMA
\paragraph{Bilinear and Multilinear Algebra}
Bilinear maps.  Quadratic forms.
Orthogonal projection
Gram-Schmidt.
Singular values.  Further factorizations.
Multilinear algebra: vectors, covectors, bivectors, tensors, symmetric
tensors, anti-symmetric tensors.  Contractions.
Extension of a metric to all tensorial layers.
{LA->BMA}

%^ CSV
\paragraph{Calculus of Several Variables.}
Euclidean spaces: identities and inequalities.
Basic topology: limits, compactness, completeness, continuity.
Continuous and bounded functions.
Definition of the derivative and its main properties.
Chain rule.
Jacobian matrix and partial derivatives.
Directional derivative.
Higher-order derivatives, and Taylor theorem.
Inverse function theorems.
Implicit function theorem.
Integration.  Measure zero and content zero.
Fubini theorem.  Change of variable theorem.
[Spivak's Calculus on Manifolds, first 74 pages]
{C1V->CSV}


%^ VC
\paragraph{Vector Calculus}
(1) scalar and vector fields:
Scalar and vector fields in the plane and in space.
Gradient, divergence, curl.
Line integrals, region integrals, flows and fluxes.
Green, Gauss and Stokes theorems.
(2) geometry of curves and surfaces:
Curves in the plane (curvature, total curvature, four-vertex theorem).
Curves in space (curvature, torsion, Frenet frame).
Surfaces in space (principal directions, fundamental forms, curvatures).
Gauss-Bonnet theorem.
Abstract surfaces.
[do Carmo]
{CSV->VC, LA->VC}

%^ DG
\paragraph{Differential Geometry}
Topological manifolds.
Differential manifolds.
Submanifolds, functions, germs.
Differential forms.
Exterior derivative.
Stokes theorem.
Pfaffian forms and Frobenius theorem.
Riemannian manifolds.  Geodesics, exponential map.
Connections, parallel transport, volume form.
Local isometries.  Lie groups.
Other structures: semi-riemannian, subriemannian, symplectic, kahler, complex.
[Spivak's treatise; Berger's panoramic overview]
{GET->DG, VC->DG, BMA->DG}

%^ DT
\paragraph{Differential Topology}
Regular values, critical values, singular points, Morse lemma.
Fundamental theorem of algebra.
Morse-Sard theorem.
Brouwer fixed-point theorem.
Smooth homotopy, smooth isotopy, degree modulo 2.
Poincaré-Hopf theorem on vector fields.
Cobordism, framing, Pontryagin construction, Hopf theorem.
Classification of curves and surfaces.
[Milnor]
{CSV->DT}

%^ SG
\paragraph{Spectral geometry}
Spectrum on a graph.
Laplace-Beltrami spectrum.
Geodesic length spectrum.
Cheeger sets and cuts.
[Colin de Verdière, Berger]


%^ CO
\paragraph{Convex Optimization}
Basic optimization methods in dimension 1.
Optimization in finite dimensions, KKT.
Farkas lemma.
Legendre-Fenchel Duality.
Discrete optimization: simplex method, interior point methods.
Optimization with constraints.
Optimization vs. solution of equations.
[Rockafellar]
{VC->CO, BMA->CO}

%^ NO
\paragraph{Nonconvex Optimization}
Multi-scale, relaxation, linearization.
Derivative-free optimization methods.
Contrast-invariant optimization methods.
{VC->NO, BMA->NO}

%^ MT
\paragraph{Measure and integration}
Axiomatics of measure spaces.
Measure and cardinal.
Measure and category.
Lebesgue Integration.
Lebesgue-Stieltjes integration.
Riesz-Nagy integration axiomatics.
{CSV->MT}

%^ RA
\paragraph{Real analysis}
Constructions of real numbers: Tarski's axioms, decimal sequences, Dedekind
cuts, Cauchy sequences in Q, hyperreal numbers, surreal numbers, almost-equal
sequences in Z.
Summation of divergent series.
Punctual convergence of Fourier series.
Arzela-Ascoli.  Baire.  Cantor functions.
Decomposition f = absolutely continuous + jumps + cantor
[Rudin]
{CSV->RA}

%^ VC
\paragraph{Variational calculus}
The direct method in all combinations of dimensions.
Examples.



%^ FA
\paragraph{Functional Analysis}
(1) Theory:
Topological vector spaces.  Families of seminorms.  Hahn-Banach theorem.
Banach spaces.
Hilbert spaces.  Lax-Milgram theorem.
(2) Examples:
Semicontinuous functions.
Lp spaces.
Sobolev spaces in dimension 1.
Sobolev spaces in dimension N.
Spaces of test functions.
Spaces of distributions.
[Brézis]

%^ SA
\paragraph{Spectral analysis}
Compact operators.
Spectral theorem on Banach spaces.
Hilbert-Schmidt operators.
Spectral theorem.  Functional calculus.
[Brézis, Yosida]

%^ ODE
\paragraph{Ordinary Differential Equations}
Examples of ODE from physics and modeling.
Techniques of solution of ODE.
Geometric Interpretation of linear ODE in the plane, classification of
singular points.
Theorem of existence, uniqueness and regularity.
Sturm-Liouville problems.
[Coddington-Levinson]
{VC->ODE}

%^ PDE
\paragraph{Partial Differential Equations}
Examples of linear PDE: transport, Laplace, Poisson, Heat, Wave, Schrodinger.
Examples of nonlinear PDE: eikonal, hamilton-jacobi.
Example of PDE with no solutions / solutions with limited domain / shocks.
Solution of first-order PDE by the method of characteristics.
Classification of second-order PDE.  Examples of explicit solutions on easy
domains.
Transform methods for linear PDE.
Variational PDE.
Weak solutions using weak derivatives.
Weak solutions in the viscosity sense.
[Evans]
{ODE->PDE}

%^ DS
\paragraph{Dynamical Systems}
Feigenbaum, Lorentz, Julia, Mandelbrot.
Poincaré maps.  Structural stability.
[Strogatz]
{ODE->DS}

%^ HA
\paragraph{Harmonic Analysis}
(0) Recall of basic constructions in Fourier series and transforms.
(1) Sampling theory in euclidean space, distribution theory, Dirac combs and
Poisson duality
(2) Commutative harmonic analysis: localy compact abelian groups, haar
measure, Pontryagin duality, Schwartz spaces
(3) Compact non-commutative harmonic analysis: Paley-Wiener theorem,
representation on semisimple, solvable, groups.
(4) Harmonic analysis on homogeneous spaces, spherical harmonics.
(5) Harmonic analysis on compact manifolds, with and without boundary.
[Pontryagin, Helgason, Terras, Colin de Verdière]


%^ RT
\paragraph{Representation theory}

%^ NSA
\paragraph{Nonstandard and Synthetic Analysis}
Nonstandard analysis using dual numbers.
Nonstandard analysis using infinitesimals.
Constructive real analysis.

%^ F
\paragraph{Formalization}
Zero, first and second order logic.
Naive set theory.
Zermelo-Fraenkel set theory.
Axiom of choice and equivalents.
Inductivism and constructivism.
Intuitionistic logic.
Category theory.
Toposes and Topoi.
Bayesianism.

%^ HM
\paragraph{History of mathematics}
Basically Kline and Stilwell.
China.  Babylon.  Egypt.  Greece.  Rome.  India.  Arabia.
Europe in the middle ages.
Renaissance.  Modern times, separation of math from philosophy.
The eighteenth.  The explosion of the nineteenths.
Separation of math and physics.
Twentieth century.  Attempts at formalization of physics.
Maths and computer science.  Neural networks.

%^ A
\paragraph{Arithmetic}
Euclidean algorithm.
Prime numbers. Euler's totient.
Famous Diophantine equations.
Quadratic forms.
Mobius inversion formula.
[Hardy]

%^ AA
\paragraph{Abstract Algebra}
Basic structures: magma, semigroup, group, ring, field, module, algebra,
vector space.
Semigroups (boring, very fast, just talk about free \& free+relations).
Group theory: subgroups, normal groups, quotient groups, product, semidirect
product, group morphisms, exact sequences, finite groups, p-sylows, butterfly
lemma, classification
Rings: subrings, ideals, quotients, maximal principal and prime ideals,
domains, fields, ring morphisms, factorization.
Fields: sub-fields, field extensions, examples of number fields.
Galois theory: field decomposition, etc.
[Lang, Stewart]

%^ CA
\paragraph{Commutative Algebra}
(basically atiyah mcdonald)
Intro: rings of functions over M as a means to study the object M.
Ideals, nilpotents, radicals, local rings.
Noetherian rings.  Hilbert basis theorem.
Nullstellensatz.
Discrete valuation rings.
[Atiyah-Macdonald]

%^ ANT
\paragraph{Algebraic Number Theory}
Extensions of Q.
Elliptic curves.
Algebraic curves.
[Silverman-Tate]


%^ NNT
\paragraph{Analytic Number Theory}
Euler, Bernoulli, Gauss, Jacobi, Abel, Ramanujan.
Zeta functions.
[Apostol]

%^ CNT
\paragraph{Computational Number Theory}
Seminumerical algorithms.  Floating point, integers, rationals, polynomials.
Finite fields.
Elliptic curves.
Hyperelliptic curves and function fields.
[Cohen]

%^ C
\paragraph{Combinatorics}
Enumerative combinatorics.
Generatingfunctionology.
Statistical combinatorics.  Ramsey.


%^ P
\paragraph{Programming}
Basically, be able to write a list of simple programs.
Hello world.  Convert Celsius to Fahrenheit.
Compute area of arbitrary triangle.
Sorting algorithms.  String processing.
Implicit data structures: heap, disjoint set forest.
[Kernighan-Ritchie]
{}

%^ KNU
\paragraph{Algorithms}
Semirandom numbers.  Generation and statistic evaluation.
Basic overview of complexity notations.
Fundamental algorithms and data structures: sorting, searching, trees, hash
tables, maps, dictionaries, etc.
Dual numbers, automatic differentiation.
[Knuth, Sedgewick]
{P->KNU}

%^ CT
\paragraph{Computation theory}
Finite automata. Nondeterministic automata. Regular expressions.
Turing machines.  Lambda calculus.  Futamura projections and combinators.
Computability.  Gödel theorem.
Computational complexity.
{KNU->CT}

%^ G
\paragraph{Graphics}
Rendering of simple polyhedral figures.  Light models.
3D geometry, change of bases.  Projective geometry.
Exercices with modern opengl.
{P->G}

%^ NM
\paragraph{Numerical Methods}
Comprehensive introduction to floating point numbers.
Polynomial interpolation.
Polynomial approximation.
Numerical derivatives and integrals.
Order of an integration method.
Practical integration in higher dimensions.
Numerical solution of equations.
Iterative techniques for local optimization.
{P->NM, CSV->NM}

%^ NLA
\paragraph{Numerical Linear Algebra}
Matrix analysis: matrix norms, induced norms, spectral radius, householder
theorem, Gerschgörig theorem, matrix conditioning
Algebraic matrix factorization: triangular systems, LU, pivots, cholesky
Iterative solution of linear systems: Jacobi, Gauss-Seidel, gradient descent,
conjugate gradients.
Eigen and Singular decompositions: power method, inverse power method, QR,
SVD
Applications: quadratic minimization, linear regression, discrete linear ODE
and PDE, splines, polynomial factorization, oscillation modes
{P->NLA, BMA->NLA}

%^ NOE
\paragraph{Numerical ODE}
Visualization of vector field integration.
Euler explicit, Euler implicit, Runge-Kutta
perturbation analysis, numerical control,
multiple-scale analysis, kalman filtering

%^ NPE
\paragraph{Numerical PDE}
Many examples of PDE discretization via sparse matrices.
Stability, Consistency and Convergence.
Graph-discretization of geometric PDE.
Finite elements.
Finite volumes.
Discretization of variational problems.
Primal-dual methods for variational problems.
Fast-marching methods for first-order problems: transport, burgers, etc.
Fourier, multiscale, multigrid for elliptic problems.
Free discontinuity problems.

%^ NUO
\paragraph{Numerical Optimization}
Automatic differentiation (forward and backward).
Levenberg-Marcquardt, Newton and semi-newton methods.
Gradient descent, Randomized gradient descent.

%^ GRT
\paragraph{Graph Theory}
Graph combinatorics.
Graph algorithms, paths partitions and flows.
Algebraic graph theory, spectral graph analysis.
Statistical graph theory, percolation.
Vector calculus and morphology on graphs.
Exterior calculus on combinatorial cell complexes.
Graphs and groups: the Cayley diagram.

%^ GS
\paragraph{Synthetic Geometry}
Euclidean constructions in the plane.
Conic sections.
Non-euclidean constructions in the plane.
Spherical geometry.

%^ GA
\paragraph{Abstract Geometry}
Ordered Geometry.
Affine Geometry.
Euclidean Geometry.
Projective Geometry.
Conics and Quadrics.
Discrete symmetry groups.
Polyhedra and polytopes.
[Coxeter]
{}

%^ GET
\paragraph{General Topology}
Metric spaces: balls, open sets, continuous functions
Topological spaces.  Continuous functions.
Connectedness.  Compactness.  Separation.
Homotopy of continuous functions.
{CSV->GET}

%^ AT
\paragraph{Algebraic Topology}
Polyhedra and cell complexes.
Simplicial homology: chains, complexes, morphisms of complexes, homotopies
between morphisms.
Singular homology.  H0, H1, fundamental group, Mayer-Vietoris sequences.
Axiomatic homology.
Topological surfaces, surface classification.  Topological manifolds.
Relationship with differential structures: Betti, Cech cohomology, etc.

%^ AG
\paragraph{Algebraic Geometry}
Algebraic varieties.
[Fulton, Serre, Hartshorne]

%^ ARG
\paragraph{Arithmetic Geometry}
Class field theory and so on.


%^ DP
\paragraph{Discrete Probability}
Kolmogorov axiomatics using measure theory.
Cox axiomatics using probable inference.
Basic examples of probabilistic computations, conditional probability, bayes
theorem.
Random variables, events, expectation, entropy, variance.
Some famous discrete distributions.
Characteristic functions, generating functions, moments.
[Cox, Baldi]

%^ CP
\paragraph{Continuous Probability}
Probability densities, moments, characteristic functions.
Densities vs. random variables.
Modes of convergence of random variables.
Examples of distributions.
Probability distributions in N-dimensions.
Central limit theorem.
Non-finite variance.  Stable distributions.  Kolmogorov-Gnedenko central
limit theorem.
Large deviation theory.
Extreme value analysis.
[Chung?]


%^ I
\paragraph{Inference}
The interpretation of probability.
Basic examples of bayesian inference.
Marginalization.
Uninformative priors, invariant priors, conjugate priors.
[MacKay, Jaynes]



%^ IT
\paragraph{Information Theory}

%^ SIP
\paragraph {Signal Processing}
Analog one-dimensional signal processing.
Digital signal processing.
Modulation.
Radar focusing.

%^ SP
\paragraph{Stochastic processes}

%^ GAT
\paragraph{Game theory}

%^ ML
\paragraph{Machine Learning}



%^ GP
\paragraph{General Physics}
[Feynman]

%^ CLM
\paragraph{Classical Mechanics}
Newtonian mechanics (curves on Euclidean space).
Lagrangian mechanics (geodesics on a Riemannian manifold).
Hamiltonian mechanics (PDE on a Symplectic manifold).
Noether theorem (symmetries).
Canonical equations.
[Landau, Lanczos, Arnold]

%^ MAX
\paragraph{Electromagnetism}
Maxwell equations in empty space, electromagnetic waves.
Maxwell equations with carges and currents.
Macroscopic maxwell equations.
Movement of a charged particle on an electromagnetic field.
Self-interacting charges and currents.
Geometric rewriting of Maxwell equations.
Distributional rewriting of Maxwell equations.

%^ FM
\paragraph{Fluid Mechanics}

%^ CFT
\paragraph{Classical Field Theory}
Newtonian dust. ...

%^ SR
\paragraph{Special Relativity}

%^ SM
\paragraph{Statistical Mechanics}

%^ QM
\paragraph{Quantum Mechanics}
Quantum Electro Dynamics.
[Feynman, Mackey]

%^ GR
\paragraph{General Relativity}
[MTW, Hawking-Ellis]

%^ QFT
\paragraph{Quantum Field Theory}
Twelve particles of matter.
Four forces of nature.

%^ BI
\paragraph{Bioinformatics}
Unknown elaborate constructions that lead to the possibility of ADN
sequencing.

%^ CY
\paragraph{Crystallography}
Diffraction.  The molecule of pennicilin.  General statement of the
crystallographic problem.  Modern solutions using regularization and
sparsity.  Dust crystallography.

%^ CEM
\paragraph{Celestial mechanics}
The two-body problem, basic orbit determination.  The n-body problem, general
theorems.  Restricted three-body problems, some periodic orbits, some
particular points.  Piecewise elliptic approximation.  Halo orbits.
The orbit of a satellite in practice.

%^ O
\paragraph{Optics}
Geometric optics of light rays.
Geometric optics of wavefronts.
Fresnel and Fraunhoffer diffraction.
Optical aberrations.
Design of achromatic lenses.
Astronomical seeing.


% vim:set tw=77 filetype=tex spell spelllang=en:
