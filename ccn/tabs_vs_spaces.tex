\title{Tabs vs Spaces}

Tabs vs spaces is my favorite holy war.  I have managed to hold a set of
convictions that is maximally contradictory with the two major lines of
thought.

These are the rules for indenting source code of programs:

\begin{enumerate}
	\item In most programming languages, you can use either tabs or
		spaces for indentation.
	\item If you use spaces, it must be 8 spaces per indentation level.
	\item Do not mix tabs and spaces for indentation on the same file.
	\item In python, you can only use tabs.
\end{enumerate}

Rationale (1):

I do not care what happens when I press the TAB key on my text editor, as I
never do that.  Any modern text editor supports auto-indentation, where
blocks of code are automatically indented according to the language
semantics.  For example, if you have the following text in python (X
indicates the position of the cursor)

\begin{verbatim}
        if (a > 0):X
\end{verbatim}

and you press ENTER, then you see this:

\begin{verbatim}
        if (a > 0):
                X
\end{verbatim}

Then you write the code of your block, which may have several lines, and when
you are done you press BACKSPACE for exiting the block.
Ultimately, the file you are editing is stored with either tabs or spaces,
and you do not really care about that.


\bigskip

Rationale (2):




% vim:set tw=77 filetype=tex spell spelllang=en:
