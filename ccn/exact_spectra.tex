\title{Shapes with closed-form eigenmodes}

\newcommand{\R}{\mathbf{R}}
\newcommand{\Z}{\mathbf{Z}}
\newcommand{\Q}{\mathbf{Q}}
\newcommand{\C}{\mathbf{C}}
\newcommand{\U}{\mathbf{U}}
\newcommand{\T}{\mathbf{T}}
\newcommand{\ud}{\mathrm{d}}
\newcommand{\x}{\mathbf{x}}


This is a list of compact shapes whose Laplace-Beltrami
eigenfunctions can be expressed in terms of special functions.  It is
based mostly on the review articles by Grebenkov.

There are a just a few shapes where Helmholtz equation admits
closed-form solutions:

\begin{enumerate}
	\item The one-dimensional interval, using trigonometric functions.
	\item The disk, using Bessel functions
	\item The ellipse, using Mathieu functions
%	\item The surface of the sphere, using spherical harmonics (which
%		are polinomials of trigonometric functions)
	\item The equilateral triangle, using Lamé functions
\end{enumerate}

Other shapes can be also treated from these five ``basic'' types:

\begin{enumerate}
		\setcounter{enumi}{4}
	\item The rectangle in~$\R^d$, which is a separable product of
		intervals
	\item The cylinder, which is the product of a disk and an interval
	\item The right isosceles triangle, which is half a square and its
		eigenfunctions are obtained as those that vanish on the
		diagonal of the square
	\item The $90-60-30$ triangle, which is half an equilateral
		triangle
	\item Other separable products or symmetric subsets of treatable
		shapes
\end{enumerate}

Here we treat only the ``hard'' case of shapes with boundary.  The
case of compact manifolds without boundary
(circle, torus, sphere, ...) is somewhat easier.

\clearpage
\section{General considerations}

The basic result is the following.

\begin{proposition}
	Let~$\Omega$ be a compact manifold with boundary~$\partial\Omega$
	with Laplace-Beltrami operator~$\Delta_\Omega$.  Helmholtz
	equation with Dirichlet boundary conditions
	\begin{equation}\label{eq:helmholtz}
		\begin{cases}
			\Delta_\Omega u = -\lambda u & \Omega \\
			u = 0 & \partial\Omega
		\end{cases}
	\end{equation}
	has a numerable sequence of
	solutions~$\Delta_\Omega\varphi_n=-\lambda_n\varphi_n$,
	for~$n=1,2,3,\ldots$.  The numbers~$\{\lambda_n\}$, called
	the~\emph{spectrum} of~$\/\Omega$, are strictly positive and go to
	infinity.  The functions~$\varphi_n$, called the~\emph{normal
	modes}, are~$\mathcal{C}^{\infty}(\Omega)$ and form a Hilbert
	basis of~$L^2(\Omega)$.
\end{proposition}

This standard result (see for example Warner--Foundations of Differentiable
Manifolds an Lie Groups) settles the whole question of Helmholtz
equation, at least from the theoretical point of view.  In practice,
we may want to compute the spectrum or the normal modes numerically.
This can be done in general, by representing the shape~$\Omega$ by a
fine simplicial mesh, so that the equation in a continuous domain is
transformed as a linear algebra eigenvalue problem in finite
dimension.  Since the corresponding matrix is symmetric and positive
definite, the problem is particularly well posed.  But solving the
discretized problem exactly does not mean that we are close to the
real solution.  How fine must be the mesh so that the high-frequency
modes are correctly represented?  For that, it is very useful to have
a few simple domains where the solutions can be expressed explicitly
in terms of known functions.

The technique for solving equation~(\ref{eq:helmholtz}) is often
similar.  First a family of solutions of~$\Delta u = -\lambda u$ is
determined.  This family depends on a few continuous parameters.  It
is then found that for some particular, discrete, values of these
parameters, the boundary condition is magically satisfied.  These
parameters are called the ``characteristic values''.

The ideal case happens in the unit interval~$\Omega=[0,1]$.  The
equation without boundary conditions is~$u''=-\lambda u$, whose
solutions are of the general form~$\alpha\sin(\sqrt{\lambda}x)+
\beta\cos(\sqrt{\lambda}x)$.  By imposing the boundary conditions we
find that~$\beta=0$ and that~$\sqrt{\lambda}$ must be an integer
multiple of~$\pi$.


\clearpage
\section{The interval}

Let~$\Omega=[0,1]$.  The equation without boundary conditions is
\begin{equation}\label{eq:sincos}
	u''=-\lambda u
\end{equation}
solutions of this equation for an arbitrary~$\lambda\in\C$ are of the form
\[
	u(x)=
	\alpha_1 e^{\mu x}
	+
	\alpha_2 e^{-\mu x}
	+
	\alpha_3 e^{i\mu x}
	+
	\alpha_4 e^{-i\mu x}
\]
where~$\mu^2=-\lambda$.  For~$\lambda>0$ and real-valued~$u$, this
gives~$\alpha_1=\alpha_2=0$.  Thus  we are interested in solutions
of~(\ref{eq:sincos}) of the form
\[
	u(x)=\alpha\sin\left(\sqrt{\lambda}
	x\right)+\beta\cos\left(\sqrt{\lambda}x\right).
\]
Now, the boundary conditions are~$u(0)=0$ and~$u(1)=0$.  The first
boundary condition implies that~$\beta=0$.  The second boundary
condition now says
\begin{equation}\label{eq:sinroots}
	\sin\left(\sqrt{\lambda}\right) = 0
\end{equation}
Since the sine function has roots at integer multiples of~$\pi$, the
solutions of this equation are the numbers~$\lambda_n=n^2\pi^2$.

The computation above proves that, for~$\Omega=[0,1]$, the
spectrum and its corresponding normal modes are
\[
	\lambda_n = \pi^2n^2
	\qquad n=1,2,3,\ldots
\]
\[
	\varphi_n(x) = \sin\left(\pi n x\right)
	\qquad n=1,2,3,\ldots
\]
On an interval of arbitrary length~$\Omega=[0,L]$ the result is
obtained by a linear change of variable
\[
	\varphi_n(x) = \sin\frac{\pi n x}L
	\qquad n=1,2,3,\ldots
\]
\[
	\lambda_n = \frac{\pi^2n^2}{L^2}
	\qquad n=1,2,3,\ldots
\]


\clearpage
\section{The rectangle}

Consider now~$\Omega=[0,A]\times[0,B]\subseteq\R^2$.
The Laplacian in two variables is separable: if~$u(x,y)=p(x)q(y)$
then~$\Delta u = p''q + q''p$.  Thus we can obtain solutions of
the two-dimensional Helmholtz equation by solving separately
the two equations
\[
	p''=-\lambda p
	\qquad\qquad
	q''=-\lambda q
\]
and then building~$u=pq$.  Notice that this will produce~\emph{some}
solutions of Helmholtz equation, but of course not all of them (in
particular, the property of being separable is not even closed by
linear combinations!).  Some kind of completeness of these solutions
will be needed to to proven later, after imposing the boundary
conditions.

In the case of the rectangle, the boundary conditions can be imposed
separately for~$p$ and~$q$.  Notice that the functions
\[
	\varphi_{m,n}(x,y)=\sin\frac{\pi m x}A\cos\frac{\pi n y}B
\]
satisfy the boundary conditions and satisfy Helmoltz equation with
eigenvalues
\[
	\lambda_{m,n} =
	\frac{\pi^2m^2}{A^2}
	+
	\frac{\pi^2n^2}{B^2}.
\]
The case of the rectangle is particularly simple, but it has a
similar
structure than the cases for the disk and the ellipse.
The difference is that the condition of separability is different.




\clearpage
\section{The disk}

Consider the unit disk~$\Omega=\left\{\x\in\R^2\ :\
\|\x\|\le1\right\}$.  This set is not separable in Cartesian
coordinates~$\x=(x,y)$, but it is separable in polar
coordinates~$(r,\theta)$, defined by
\[
	\begin{cases}
		x = r\cos\theta \\
		y = r\sin\theta
	\end{cases}
	\qquad\qquad
	\qquad
	\begin{cases}
		r = \sqrt{x^2+y^2} \\
		\theta = \mathrm{atan2}(y,x)
	\end{cases}
\]
to write Helmholtz equation in polar coordinates, we need to
express the Laplacian in terms of derivatives of~$r$ and~$\theta$:
\[
	\Delta
	\ = \ %
	\frac{\partial^2}{\partial x^2}
	+
	\frac{\partial^2}{\partial y^2}
	\ = \ %
	\frac{\partial^2}{\partial r^2}
	+
	\frac 1r
	\frac{\partial}{\partial r}
	+
	\frac 1{r^2}
	\frac{\partial^2}{\partial \theta^2}.
\]
Now we can apply separation of variables:
if~$u(r,\theta)=f(r)g(\theta)$ then Helmholtz equation becomes
\[%\Delta u =
f''(r)g(\theta)+\frac{f'(r)}rg(\theta)+\frac{f(r)}{r^2}g''(\theta)
=-\lambda f(r)g(\theta)
\]
or, rearranging the terms,
\[
	\left(
f''(r)
+
\frac{f'(r)}r
+\lambda f(r)
\right)g(\theta)
=
-
\frac{f(r)}{r^2}g''(\theta)
\]
This equation will hold whenever the following two do, for the same
``characteristic constant''~$\mu$:
\[
	\begin{cases}
		g''=-\mu g \\
		f''+{f'}/{r}+\lambda f = -\mu{f}/{r^2}
	\end{cases}
\]
These are now two separate ordinary equations on the
variables~$\theta$ and~$r$, respectively.  Let us consider them
separately.

The first equation~$g''=-\mu g$ is an old friend.  Its solutions are
sines and cosines of~$\sqrt{\mu}\theta$.  Now, if we want that this
represents something in polar coordinates, then~$g$ must be~$2\pi$
periodic, so~$\sqrt{\mu}$ must be an integer multiple of~$2\pi$,
thus~$\mu_k=4\pi^2 k^2$ and we have two families of solutions
for~$g$:
\[
	g^0_k(\theta) = \cos(2\pi k \theta)
	\quad k=0,1,2,\ldots
\]
\[
	g^1_k(\theta) = \sin(2\pi k \theta)
	\quad k=1,2,3,\ldots
\]

Now consider the second equation
\[
	r^2f''(r) + rf'(r)
	+\left(\lambda r^2+
	%\left(\frac{2\pi k}{r}\right)^2
	4\pi^2 k^2
	\right)f(r) = 0
\]
This is called Bessel equation, and its solution are the Bessel
functions

\section{The ellipse}

\section{The equilateral triangle}

% vim:set ts=3 sw=3 tw=69 filetype=tex spell spelllang=en:
