
\newcommand{\ud}{\mathrm{d}}
\newcommand{\R}{\mathbf{R}}

\newcommand{\x}{\mathbf{x}}
\newcommand{\y}{\mathbf{y}}
\newcommand{\p}{\mathbf{p}}

\newcommand{\1}{\textbf{1}}

% \def\R{\mathbf{R}}
% \def\C{\mathbf{C}}
% \def\N{\mathbf{N}}
% \def\T{\mathbf{T}}
% \def\x{\mathbf{x}}
% \def\a{\mathbf{a}}
% \def\y{\mathbf{y}}
% \def\m{\mathbf{m}}
% \def\b{\mathbf{b}}
% \def\u{\mathbf{u}}
% \def\Z{\mathbf{Z}}
 \def\R{\textbf{R}}
 \def\C{\textbf{C}}
 \def\N{\textbf{N}}
 \def\T{\textbf{T}}
 \def\x{\textbf{x}}
 \def\a{\textbf{a}}
 \def\y{\textbf{y}}
 \def\m{\textbf{m}}
 \def\b{\textbf{b}}
 \def\u{\textbf{u}}
 \def\Z{\textbf{Z}}

\def\F{\mathcal{F}}
\def\d{\mathrm{d}}
%\DeclareMathOperator*{\argmin}{arg\,min}
%\DeclareMathOperator*{\argmax}{arg\,max}
\newcommand{\med}[0]{\mathrm{med}}
\newcommand{\reference}[1] {{\scriptsize \color{gray}{#1}}}
\newcommand{\referenceq}[1] {{\tiny \color{gray}[{#1}]}}
\newcommand{\referencep}[1] {{\tiny \color{gray}  #1 }}
\newcommand{\unit}[1] {{\tiny \color{gray}  #1 }}

% \parens{x}      ->  (x)
% \pairing{x}{y}  ->  <x,y>
\newcommand{\parens}[1]{\left(#1\right)} % (x)
\newcommand{\pairing}[2]{\left\langle #1,\,#2\right\rangle} % <x,y>

% \abs{x}         ->    |x|
% \Abs{x}         ->   ||x||
% \ABS{x}         ->  |||x|||
\newcommand{\abs}[1]{\left|#1\right|}
\newcommand{\Abs}[1]{\left\|#1\right\|}
\newcommand{\ABS}[1]{{\left\vert\kern-0.25ex\left\vert\kern-0.25ex\left\vert #1 \right\vert\kern-0.25ex\right\vert\kern-0.25ex\right\vert}}



\title{Algorithms for the geometric median}


\section{The geometric median}

\begin{definition}[Geometric median]
Let
\(
	\left\{\a_1,\ldots,\a_n\right\}\subseteq\R^d
\)
be~$n$ points in~$\R^d$.
A~\emph{geometric median} of these points is any point~$\x\in\R^d$ minimizing
the sum of distances:
\begin{equation}\label{eq:objective}
	%\med(\a_1,\ldots,\a_n) = \arg
	\min_{\x\in\R^d} f(\x)
	\qquad
	\qquad
	\textrm{where}
	\qquad
	\qquad
		%\displaystyle
	f(\x) = \sum_{i=1}^n \left\|\x-\a_i\right\|
\end{equation}
\end{definition}

\begin{proposition}
	Any finite non-empty set of points has at least one geometrical
	median.
\end{proposition}

\begin{proof}
	The function~$f(\x)$ in~\ref{eq:objective} is continuous, convex and
	has compact level sets.
\end{proof}

\begin{proposition}\label{eq:uniq}
If the points
\(
	\left\{\a_1,\ldots,\a_n\right\}\subseteq\R^d
\)
are not aligned then their geometrical median is unique.
\end{proposition}

Before proving proposition~\ref{eq:uniq} let us recall some simple cases.

\begin{proposition}[median in dimension 1]
Let~\( \left\{a_1,\ldots,a_n\right\}\subseteq\R \)
and let~$a_{(1)}\le a_{(2)}\le\cdots\le a_{(n)}$
be the same numbers in
ascending order.  In the odd case~$n=2k-1$ the median is~$a_{(k)}$.  In the
even case~$n=2k$ the median is any number in the
interval~$\left[a_{(k)},a_{(k+1)}\right]$.
\end{proposition}

\begin{proof}
	The function~$f(x)=\sum \abs{x-a_i}$ is piecewise affine with
	%odd or even
	integer slopes.
\end{proof}

\begin{proposition}[median of~$3$ points in~$\R^2$]
	Let~$\a_1,\a_2,\a_3$ be three non-aligned points in~$\R^2$, forming a
	triangle.  If the triangle has an angle larger or equal
	than~$120^\circ$, then the median is the vertex of this angle.
	Otherwise, %the three angles are smaller than~$120^\circ$ and
	the median is the Fermat point of the triangle, that sees the three
	successive points in egual angles of~$120^\circ$.
\end{proposition}

\begin{proposition}[median of~$4$ points in~$\R^2$]
	Let~$\a_1,\a_2,\a_3,\a_4$ be four non-aligned points in~$\R^2$.  If the
	convex hull of the four points is a triangle, the median is the point
	which is not a vertex.  Otherwise, the convex hull is a quadrilateral
	and the median is the point where its diagonals cross.
\end{proposition}

median of 5 points (bajaj theorem)

gradient and hessian of the objective function

strict positive definiteness of the hessian

\section{Weiszfeld algorithm}

three views of weiszfeld algorithm

as an iteratively reweighted least squares

as fixed point iterations

as a gradient descent

first problem: not well-defined

second problem: slow convergence


\section{Other algorithms}

brute force minimization on the~$a_i$, hoping that it will be close to the
optimim

gradient descent with larger steps

newton method on the region of convergence

newton-armijo

bfgs

low-dimensional hessians

stochastic gradient descent

sampling-based methods

coresets




