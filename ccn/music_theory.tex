\title{Math notes in music theory}

These are some personal notes on the (mathematical) theory of music.
The required knowledge is single-variable calculus and Fourier analysis.

Overview: A musical piece is a single sound wave.  This sound wave can be
analyzed at different levels.  The four levels, in increasing order of
complexity are timbre, harmony, rhythm and melody.

\section{Timbre}

If you play the same song in different instruments it sounds
different.  This difference is called~\emph{timbre}.
%The timbre is the most important aspect in music theory, for it will
%determine the harmony and the melody.
Let us give a mathematical formalization of timbre.
A sound wave is a function~$f(t)$ that describes the evolution of air
pressure along the time~$t$.  To define the timbre of~$f(t)$ we assume first
that the sound is in a stationary regime: thus we ignore the attack of the note
and its decay over time; we will review this simplification later.  This means
that the sound wave can be expressed in the following form
\begin{equation}\label{eq:soundwave}
	f(t) = \sum_{n\ge 0} a_n\sin\left(\omega_n t\right)
\end{equation}
where~$a_n>0$ and~$\omega_n>0$ for~$n=0,1,2,\ldots$.  For all practical
purposes we can assume that this sum has a finite number of terms.  The
number~$\omega_0$ is called the~\emph{fundamental frequency} or
the~\emph{pitch} of~$f(t)$, and the numbers~$\mu_n=\dfrac{\omega_n}{\omega_0}$
are called the~\emph{overtones} of the sound~$f(t)$.  The numbers~$a_n$ are
called the~\emph{amplitudes} and the numbers~$\dfrac{a_n}{a_0}$ are
the~\emph{relative amplitudes} of the sound~$f(t)$.  Either~$a_0$
or~$\sqrt{\sum {a_n}^2}$ is called the~\emph{volume} of the
sound.  For many typical sounds the sequence of amplitudes is
non-increasing~$a_n\ge a_{n+1}$ and it can be normalized by setting~$a_0=1$.

This definition warrants some commentary.  The first commentary is, the fact
that a sound wave is stationary does not mean that it is periodic.  For
example, the function~$f(t)=\sin(t)+\sin(\pi t)$ is not periodic but it is
nonetheless stationary.  Hearing the sound~$f(t)$, one perceives the
superposition of two pure waves of different frequency.  It is
indistinguishable from the sound
wave~$g(t)=\sin(t)+\sin\left(\tfrac{22}7t\right)$ which is periodic.

The second commentary, closely related to the first one, concerns the phases of
the sinusoidal waves.  We could have defined a stationary sound wave as
\begin{equation}\label{eq:soundwave-phases}
	f(t)=\sum_{n\ge 0}a_n\sin(\omega_nt+\varphi_n)
\end{equation}
or equivalently
\begin{equation}\label{eq:soundwave-cosines}
	f(t)=
	\sum_{n\ge 0}a_n\sin(\omega_nt)
	+\sum_{n\ge 1}b_n\cos(\omega_nt).
\end{equation}
Notice that the forms~\ref{eq:soundwave-cosines}
and~\ref{eq:soundwave-phases} are equivalent through elementary trigonometry.
Auditory experiences by Ohm (1843) show that the phases~$\varphi_n$
have no audible effect.  The reality is of course more subtle, and in some
cases phase distortion is indeed audible (see
\url{http://www.silcom.com/~aludwig/Phase_audibility.htm} and the references
therein), but for the moment we will assume that the observation by Ohm holds
true.  Thus, the phases~$\varphi_n$ on formula~\ref{eq:soundwave-phases} can
take arbitrary values, and for all we care we set them to~$\varphi_n=0$,
recovering formula~\ref{eq:soundwave}.

% harmonic vs inharmonic waveforms
Using a digital synthesizer one can easily produce sounds of the general
form~\ref{eq:soundwave} above.  Any particular choice of amplitudes and
partials will give a particular timbre; and that's all there is to timbre.



% typical waveforms

% decay

% attack

% phases revisited

% physical modelling of timbre

% detailed study of the bell


\section{Harmony}

% dissonance

% harmonic consonance

% chords

% chord sequences

% scales, tuning, and the circle of fifths

% 

\section{Rhythm}

Rhythm is low-frequency timbre.

\section{Melody}

Melody is the most difficult part of music theory to formalize in our
framework.



% vim:set tw=77 filetype=tex spell spelllang=en:
