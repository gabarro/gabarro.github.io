\title{Geometric Signals}



\newcommand{\1}{\mathbf{1}}
\newcommand{\R}{\mathbf{R}}
\newcommand{\T}{\mathbf{T}}

To study a geometric object~$M$, people often use regular functions
between~$M$ and spaces with well-known properties, such as~$\R^n$,~$\T^n$
or~$S^n$.  The resulting spaces of functions have an algebraic structure
whose properties correspond to geometric properties of~$M$.  Typically,
families of functions of the form~$X^n\to M$ form a sequence of groups
(homology, homotopy), and the structure of these groups and their quotients
provides a lot of geometrical information about~$M$.  On the other hand,
families of functions~$M\to X^n$ for a sequence of vector spaces
(cohomology) whose dimensions and quotients give geometrical information
about~$M$, in a different, often easier to interpret form.

The easiest example is the algebraic definition of~\emph{connexity}.  Imagine
that~$M$ has~$n$ connected components.  How can we recover~$n$ using algebra?
%
Via cohomology: consider the set of smooth functions~$M\to\R$ whose
derivative is zero.  It is a vector space of dimension~$n$.
Notice that this is the quotient space of the set of all smooth functions,
by its subspace defined by a linear condition~$df=0$.
%
Via homology: consider the

In more concrete terms, functions~$\R\to M$ are called~\emph{paths}
or~\emph{walks}, or~\emph{trajectories}.  And functions~$M\to\R$ are
called~\emph{signals}, or simply~\emph{functions}.

In this note we study geometric signal processing, that is


For
example, continuous functions~$\R\to M$ and~$\T\to M$ are respectively
called~\emph{paths} and~\emph{cycles} on~$M$.  The sets of all paths or
cycles can be endowed with the structure of a groupgk



\section{Vector calculus}

% vector calculus in the plane

% vector calculus in three-dimensional space

% integral theorems: green, gauss, stokes

% limitations of vector calculus (e.g. two types of vector fields)

% geometry of plane curves, space curves, surfaces

\section{Differential geometry}

% GRAND SCHEME OF NON-METRIC STUFF, SUBSET OF NEXT SECTION'S SCHEME

% manifolds, charts, atlases

% tangent vectors, vector fields, tangent bundle

% tensors, tensor product, contraction

% differential forms, wedge product

% exterior derivative

% lie derivatives (of functions, forms, tensors)

% lie bracket

% chains, integrals and stokes theorem

% currents and forms


\section{Riemannian geometry}

% GRAND SCHEME OF METRIC STUFF WITH METRIC-ONLY STUFF IN RED

% metric, length, energy

% geodesic equations

% musical isomorphisms, hodge duality

% laplace beltrami

% covariant derivative

% parallel transport

% killing vector fields (infinitessimal isometries)

% hessian

\section{Other structured geometries}

% symplectic structure

% complex structure

% kahlerian structure


\section{Discrete case}

