

\newcommand{\1}{\mathbf{1}}
\newcommand{\R}{\mathbf{R}}
\newcommand{\T}{\mathbf{T}}
\newcommand{\Z}{\mathbf{Z}}
\newcommand{\ud}{\mathrm{d}}
\newcommand{\ds}{\displaystyle}

% \abs{x}         ->    |x|
% \Abs{x}         ->   ||x||
% \ABS{x}         ->  |||x|||
\newcommand{\abs}[1]{\left|#1\right|}
\newcommand{\Abs}[1]{\left\|#1\right\|}
\newcommand{\ABS}[1]{{\left\vert\kern-0.25ex\left\vert\kern-0.25ex\left\vert #1 \right\vert\kern-0.25ex\right\vert\kern-0.25ex\right\vert}}

% \parens{x}      ->  (x)
% \pairing{x}{y}  ->  <x,y>
\newcommand{\parens}[1]{\left(#1\right)} % (x)
\newcommand{\pairing}[2]{\left\langle #1,\,#2\right\rangle} % <x,y>




\title{summary of mechanics}

(just some notes while I read Arnold)

\section{Introduction}

There are three formalizations of classical mechanics.

In~{\bf Newtonian mechanics}, a physical system is described by a set of~$N$
particles moving in~$\R^3$.  The state of the system is given by the
positions and the speeds of all the
particles:~$(x,\dot x)\in\R^{3N}\times\R^{3N}$.  The evolution of the system
is governed by Newton's equation
\[
	\ddot x=F(\dot x, x,t)
\]
Where~$F:\R^{3N}\times\R^{3N}\times\R$ is a function determining the system
called the~\emph{force field}.

In~{\bf Lagrangian mechanics}, a physical system is described by a point in
a differential manifold~$M$ called the~\emph{configuration space}.  The tangent
space~$TM$ is called the~\emph{state space}.  The physics is governed by a
function~$L:TM\to\R$ called the~\emph{lagrangian}.  The trajectories of the system
are curves~$q:I\to M$ that minimize the functional
\[
	E(q)=\int_I L(q(t),\dot q(t),t)\mathrm{d}t.
\]
By the direct method in the calculus of variations, these trajectories
satisfy the Euler-Lagrange equations
\[
	\frac{\partial L}{\partial q}
	-\frac{\mathrm{d}}{\mathrm{d}t}\frac{\partial L}{\partial\dot q} = 0.
\]
Newtonian mechanics whith a conservative force field~$(F=\nabla V)$
is a particular case of Lagrangian mechanics
mechanics when~$M=\R^{3N}$ and~$L=\frac12\left\|\dot q\right\|^2-V$.


In~{\bf Hamiltonian mechanics}, a physical system is described by a point in
a symplectic manifold~$\Omega$ called~\emph{phase space}.  The physics is
governed by a function~$H:\Omega\to\R$ called the~\emph{hamiltonian}.
Trajectories of the system are curves in~$\Omega$ satisfying Hamilton's
equations
\[
	\frac{\mathrm{d}q}{\mathrm{d}t}=\frac{\partial H}{\partial p},
	\qquad
	\frac{\mathrm{d}p}{\mathrm{d}t}=-\frac{\partial H}{\partial q}
\]
where~$(p,q)$ is any set of symplectic coordinates on~$\Omega$ (i.e., such
that the symplectic form is~$\omega=\mathrm{d}p_i\wedge\mathrm{d}q_i$).
Lagrangian mechanics is a particular case of Hamiltonian mechanics
when~$\Omega=T^*M$ and~$H$ is the legendre transform of~$L$.



{\bf why tho}\\
Each of the three formalisms above can be used for a complete description of
the physical world, so they are equivalent.  However, each flavor of
mechanics makes some things easier than others.

\section{Newtonian}

\subsection{Things that Newtonian mechanics does well}

\subsection{Things that Newtonian mechanics has trouble with}

\section{Lagrangian}

\subsection{Things that Lagrangian mechanics does well}

\subsection{Things that Lagrangian mechanics has trouble with}


\section{Hamiltonian}

\subsection{Things that Hamiltonian mechanics does well}

% vim:set tw=77 filetype=tex spell spelllang=en sw=2 ts=2:
