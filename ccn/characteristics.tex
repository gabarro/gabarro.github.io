
\newcommand{\ud}{\mathrm{d}}
\newcommand{\R}{\mathbf{R}}

\newcommand{\x}{\mathbf{x}}
\newcommand{\y}{\mathbf{y}}
\newcommand{\p}{\mathbf{p}}

\title{The method of characteristics}

I often forget about the following computations, and I have to spend a few hours
reconstructing them.
So here they are, in a notation that suits me.

\section{The method of characteristics in two dimensions}


%\paragraph{Statement of the problem.}
Let~$\Omega\subseteq\R^2$ be an open domain of the plane.
We want to find a function~$u:\Omega\to\R$
that satisfies the following first-order~PDE:
\begin{equation}\label{eq:pde}
		F\left(x, y, u(x,y), u_x(x,y), u_y(x,y)\right)=0
\end{equation}
where~$F(x,y,z,p,q)$ is a function~$\Omega\times\R\times\R^2\to\R$.
The equation is subject to Dirichlet boundary conditions
$u=f$ on~$\partial\Omega$
that will be treated later.

\subsection{Derivation of the characteristic ODE}
Deriving equation~\ref{eq:pde} with respect to~$x$ and~$y$ we obtain
\begin{equation}\label{eq:gradpde}
	\begin{cases}
		F_x + u_xF_z + u_{xx}F_p + u_{xy}F_q = 0\\
		F_y + u_yF_z + u_{xy}F_p + u_{yy}F_q = 0\\
	\end{cases}
\end{equation}
We will try to rewrite this by getting rid of the second derivatives.  Let us
consider a five-dimensional curve that ``traces'' a solution~$u$ in the
tangent space; i.e., a function
\begin{equation*}
	t\mapsto\left(
		x(t),y(t),z(t),p(t),q(t)
	\right)
\end{equation*}
such that
\begin{equation}
	z(t)=u\left(x(t),y(t)\right)
\end{equation}
\begin{equation}
	p(t)=u_x\left(x(t),y(t)\right)
\end{equation}
\begin{equation}
	q(t)=u_y\left(x(t),y(t)\right)
\end{equation}

Now, differentiating with respect to~$t$:
\begin{equation}\label{eq:dotz}
	\dot z=u_x\dot x + u_y\dot y
\end{equation}
\begin{equation}\label{eq:dotp}
	\dot p=u_{xx}\dot x + u_{xy}\dot y
\end{equation}
\begin{equation}\label{eq:dotq}
	\dot q=u_{yx}\dot x + u_{yy}\dot y
\end{equation}
Notice that equation~\ref{eq:dotz} can be rewritten as
\begin{equation}\label{eq:dotz2}\dot z=p\dot x+q\dot y\end{equation}

Equations~\label{eq:dotz2}, \ref{eq:dotp} and~\ref{eq:dotq} hold for an
arbitrary curve that traces the solution.  We will now impose two further
conditions on the curve to get rid of~$u$ on equations~\ref{eq:dotp}
and~\ref{eq:dotq}.  The ``magic'' conditions are the following
\begin{equation}
	\begin{cases}
		\dot x = F_p \\
		\dot y = F_q \\
	\end{cases}
\end{equation}
by combining these magic conditions with equation~\ref{eq:gradpde}, we obtain
\begin{equation}\label{eq:ode}
	\begin{cases}
		\dot x = F_p \\
		\dot y = F_q \\
		\dot z = pF_p + qF_q \\
		\dot p = -F_x - pF_z \\
		\dot q = -F_y - qF_z \\
	\end{cases}
\end{equation}
Now~\ref{eq:ode} is an ODE in~$\R^5$.  The~$\left(x(t),y(t)\right)$ solutions
of this ODE are called the~\emph{characteristic curves} of the
PDE~\ref{eq:pde}.

\subsection{Examples of characteristic equations}
%Now let us see some examples.

{\bf Linear first-order equations:}
\[
	\alpha u_x(x,y)+\beta u_y(x,y)+\gamma u(x,y)+\varphi(x,y)=0
\]
Here we have~$F(x,y,u,p,q)=\alpha p+\beta q+\gamma z+ \varphi$, thus the
characteristic equations are
\[
	\begin{cases}
		\dot x = \alpha \\
		\dot y = \beta \\
		\dot z = \alpha p + \beta q \\
		\dot p = -\varphi_x - \gamma p \\
		\dot q = -\varphi_y - \gamma q \\
	\end{cases}
\]
This means that the characteristic curves are straight lines in the
direction~$(\alpha,\beta)$.  In the common particular case when the equation
does not depend on~$u$ explicitly~$(\gamma=0)$, we have that~$z$ can be
obtained very easily, by integrating the datum~$\varphi$ along the
characteristic curves.

{\bf Eikonal equation:}
\[
	\left\|\nabla u\right\|=1
\]
Here we have~$F=\sqrt{p^2+q^2}-1$, thus the characteristic equations are
\[
	\begin{cases}
		\dot x = p\\ %/\sqrt{p^2+q^2} \\
		\dot y = q\\ %/\sqrt{p^2+q^2} \\
		\dot z = 1\\ %\sqrt{p^2+q^2} \\
		\dot p = 0 \\
		\dot q = 0 \\
	\end{cases}
\]
This can be solved explicitly: the functions~$(p,q)$ are constant,
thus~$(x,y)$ follows a straight line at speed one, and the value~$z$ of~$u$
increases linearly along these straight lines.

{\bf One-dimensional Hamilton-Jacobi equation:}
\[
	u_t=F(x,t,u,u_x)
\]
Here we have, (renaming~$t$ into~$y$):
\[
	\begin{cases}
		\dot x = -F_{u_x} \\
		\dot y = 1 \\
		\dot z = -pF_{u_x} \\
		\dot p = F_x+pF_z \\
		\dot q = -qF_z \\
	\end{cases}
\]
The second equation says that~$y$ advances at speed one (it is just the
time~$t$).  The variable~$q$ is unnecessary as it is independent to the
others.  Thus the characteristic curves are defined by:
\[
	\begin{cases}
		\dot x = -F_{u_x} \\
		\dot z = -pF_{u_x} \\
		\dot p = F_x+pF_z \\
	\end{cases}
\]

{\bf Shape-from-shading equation:}
The following equation models the image~$I(x,y)$ acquired by nadir affine
camera, when it observes a Lambertian terrain~$u(x,y)$ of albedo~$A$, lit by
a point source of light in the direction~$\left(\alpha,\beta,\gamma\right)\in
S^2$ and ambient light B:
\[
	I(x,y)=A\frac{\alpha u_x+\beta u_y+\gamma}{\sqrt{1+u_x^2+u_y^2+1}}+B
\]
formally, it is a first-order nonlinear PDE that we can solve for~$u$ to
compute the relief from the image.

By setting~$ \ J=\dfrac{I-B}A$, we have
\[
	F = \alpha p + \beta q + c - J\sqrt{1+p^2+q^2}
\]
thus the characteristic equations are
\[
	\begin{cases}
		%\dot x = \alpha - p\,J/\sqrt{1+p^2+q^2} \\
		%\dot y = \beta - q\,J/\sqrt{1+p^2+q^2} \\
		%\dot z = \alpha p + \beta q - (p^2+q^2)\,J/\sqrt{1+p^2+q^2}\\
		%\dot p = -J_x\sqrt{1+p^2+q^2} \\
		%\dot q = -J_y\sqrt{1+p^2+q^2} \\
		\dot x = \alpha - p\,J/\sqrt{\cdot} \\
		\dot y = \beta - q\,J/\sqrt{\cdot} \\
		\dot z = \alpha p + \beta q - (p^2+q^2)\,J/\sqrt{\cdot}\\
		\dot p = -J_x\sqrt{\cdot} \\
		\dot q = -J_y\sqrt{\cdot} \\
	\end{cases}
\]


\clearpage
\section{The method of characteristics in $d$ dimensions}

The computation is completely analogous to that of the plane.  The notation
seems simpler, but it requires a bit more of head-scratching to understand.
Let us consider the general first-order equation in~$\R^d$:
\begin{equation}\label{eq:pdedd}
	F(\x, u(\x), Du(\x)) = 0
\end{equation}
Where~$F(\x,u,\p)$ is a function defined
on~$\Omega\times\R\times\R^d\approx\R^{2d+1}$.  Computing the gradient of
this equation with respect to~$\x$ we obtain
\begin{equation}\label{eq:gradpdedd}
	D_{\x}F + D_{\x}u\,D_uF + D^2_{\x}u\,D_{\p}F = 0
\end{equation}
Notice that this is a~$d$-dimensional equality.  We will try to get rid of
the hessian in this equation.  Let~$u$ be a solution and pick
a~$2d+1$-dimensional curve that traces this solution in tangent space; i.e.,
a function
\begin{equation*}
	t\mapsto\left(
		\x(t),z(t),\p(t)
	\right)
\end{equation*}
such that
\begin{equation}
	z(t)=u\left(\x(t)\right)
\end{equation}
\begin{equation}
	\p(t)=D_{\x}u\left(\x(t)\right)
\end{equation}

Now, differentiating with respect to~$t$:
\begin{equation}\label{eq:dotzdd}
	\dot z=D_{\x}u\,\cdot\,\dot\x
\end{equation}
\begin{equation}\label{eq:dotpdd}
	\dot\p=D^2_{\x}u\  \dot\x
\end{equation}
Notice that equation~\ref{eq:dotzdd} can be rewritten as
\begin{equation}\label{eq:dotz2dd}\dot z=\p\cdot\dot\x\end{equation}

Equations~\ref{eq:dotz2dd} and~\ref{eq:dotpdd} hold for any
curve that traces the solution.  We will now impose~$d$ further
conditions on that curve to get rid of~$u$ on equation~\ref{eq:dotpdd}.
The ``magic'' conditions are the following
\begin{equation}
	\dot\x = D_{\p}F
\end{equation}
by combining these conditions with equation~\ref{eq:gradpdedd}, we obtain
\begin{equation}\label{eq:odedd}
	\begin{cases}
		\dot\x = D_{\p}F \\
		\dot z = \p\cdot\dot\x \\
		\dot\p = -D_{\x}F - \p F_z \\
	\end{cases}
\end{equation}
Now~\ref{eq:odedd} is an ODE in~$\R^{2d+1}$.
The solutions $\x(t)$ solutions
of this ODE are called the~\emph{characteristic curves} of the
PDE~\ref{eq:pdedd}.


% vim:set tw=77 filetype=tex spell spelllang=en:
