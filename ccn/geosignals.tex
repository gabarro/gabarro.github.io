\title{Geometric Signals}



\newcommand{\1}{\mathbf{1}}
\newcommand{\R}{\mathbf{R}}
\newcommand{\T}{\mathbf{T}}
\newcommand{\Z}{\mathbf{Z}}

To study a geometric object~$M$, people often use regular functions
between~$M$ and spaces with well-known properties, such as~$\R^n$,~$\T^n$
or~$S^n$.  The resulting spaces of functions have an algebraic structure
whose properties correspond to geometric properties of~$M$.  Typically,
families of functions of the form~$X^n\to M$ constitute a sequence of groups
(homology, homotopy), and the structure of these groups and their quotients
provides a lot of geometrical information about~$M$.  On the other hand,
families of functions~$M\to X^n$ form a sequence of vector spaces
(cohomology) whose dimensions and quotients give geometrical information
about~$M$, in a different, often easier to interpret form.

The easiest example is the algebraic definition of~\emph{connexity}.  Imagine
that~$M$ has~$n$ connected components.  How can we recover~$n$ using this kind
of algebra?
%
Via cohomology: consider the set of smooth functions~$M\to\R$ whose
derivative is zero.  This set is a vector space of dimension~$n$.
Notice that this is the quotient space of the set of all smooth functions,
modulo its subspace defined by a linear condition~$df=0$.
%
Via homology: consider the free abelian group generated by all
paths~$[0,1]^1\to M$.  The boundary of such a path is defined the difference
between its two endpoints, which is an element of the free group generated by
all points~$[0,1]^0\to M$.  The quotient of two such groups is isomorphic
to~$\Z^n$.

Functions~$\R\to M$ are thus called~\emph{paths} or~\emph{walks},
or~\emph{trajectories}.  And functions~$M\to\R$ will be called
called~\emph{signals}, or simply~\emph{functions}.

In this note we recall the main definitions of ``geometric signal
processing'' That is, we study the many operations that can be performed with
sets of functions of the form~$M\to X^n$.  We just give the definitions and
the main main results; no proofs nor lengthy interpretations.


\clearpage
\section{Vector calculus}

\subsection{vector calculus in the plane}

\subsection{vector calculus in three-dimensional space}

\subsection{integral theorems: green, gauss, stokes}

\subsection{limitations of vector calculus (e.g. two types of vector fields)}

\subsection{geometry of plane curves, space curves, surfaces}


\clearpage
\section{Differential geometry}

% GRAND SCHEME OF NON-METRIC STUFF, SUBSET OF NEXT SECTION'S SCHEME

\subsection{manifolds, charts, atlases}

The easiest way to think of a manifold is as~$\R^n$ but where most operations
are severely restricted: you cannot sum points, multiply points by scalars,
nor compute inner products.  There is no notion of translation, nor
symmetry.  What~\emph{can} you do, then?  You can only have smooth
curves~$\R\to M$ and functions~$M\to\R$, and relations between
these objects.

Formally, a differential manifold is defined as a topological space~$M$
together with a set of charts, that is, continuous bijective
functions~$\varphi:U\to\R^n$, where~$U$ are open sets of~$M$, such that for
any pair of charts~$\varphi,\varphi'$ the
function~$\varphi'\circ\varphi^{-1}$ is a smooth diffeomorphism of~$\R^n$
(nonempty if~$U\cap U'\neq\emptyset$).

An equivalent definition (but the equivalence is non-trivial) is that an
$n$-dimensional manifold is set of the form~$F^{-1}(\{0\})$
where~$F:\R^{n+k}\to\R^n$ is a smooth function such that~$0$ is not a
critical value.

A mapping~$f:M\to N$ between manifolds is said to be smooth when for any
charts~$\varphi:U\to\R^m$, $\psi:V\to\R^n$ the
mapping~$\psi\circ\varphi^{-1}:\R^m\to\R^n$ is smooth.
As particular cases  of this definition you have smooth curves and smooth
signals.

The vector space of smooth signals on~$M$ is denoted by~$C^\infty(M)$.

\subsection{tangent vectors, vector fields, tangent bundle}

There are several equivalent ways to define tangent vectors.
Let~$p$ be a point of a manifold~$M$.

(T1) A tangent vector~$X_p$ at the point~$p$ is an equivalence class of
curves~$\gamma:\R\to M$ through~$p$ (i.e., such that~$\gamma(0)=p$)
modulo the equivalence relation
\[
	\gamma_1\sim\gamma_2
	\iff
		(\varphi\circ\gamma_1)'(0)=(\varphi\circ\gamma_2)'(0)
\]
for all charts~$\varphi:U\to\R^n$ such that~$p\in U$.  Notice that this
definition makes sense because the maps~$\varphi\circ\gamma_i$ are smooth
functions~$\R\to\R$.



\subsection{tensors, tensor product, contraction}

\subsection{differential forms, wedge product}

\subsection{exterior derivative, closed forms, betti numbers}

\subsection{chains, integrals and stokes theorem}

\subsection{currents and forms}

\subsection{lie derivatives (of functions, forms, tensors)}

\subsection{lie bracket}


\clearpage
\section{Riemannian geometry}

% GRAND SCHEME OF METRIC STUFF WITH METRIC-ONLY STUFF IN RED

\subsection{metric, length, energy}

\subsection{geodesic equations}

\subsection{musical isomorphisms, hodge duality}

\subsection{laplace beltrami}

\subsection{covariant derivative}

\subsection{parallel transport}

\subsection{killing vector fields (infinitessimal isometries)}

\subsection{hessian}

\subsection{curvatures}


\clearpage
\section{Other structured geometries}

\subsection{symplectic structure}

poisson bracket, volume form, cotangent bundle

\subsection{complex structure}

\subsection{kahlerian structure}


\clearpage
\section{Discrete case}



% vim:set tw=77 filetype=tex spell spelllang=en sw=2 ts=2:
